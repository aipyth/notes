\section{Probability Measures}

The applied object in OT is a measure (probability measure).
Let's give some definitions and explain them

\begin{definition}[Measure]
  A function $\mu : \mathcal{B}(\R^{d}) \to  [0, \infty)$ is called a
  \textbf{measure} if:
  \begin{enumerate}
    \item $\mu (\varnothing) = 0$
    \item Countable additivity: $\mu \left( \bigsqcup_{i=1}^{\infty} A_i \right) = \sum_{i=1}^{\infty} \mu(A_i) $
  \end{enumerate}
\end{definition}

\begin{definition}[Probability measure]
  A function $\mu : \mathcal{B}(\R^{d}) \to  [0, 1]$ is called a
  \textbf{probability measure} if:
  \begin{enumerate}
    \item $\mu (\varnothing) = 0$
    \item Countable additivity: $\mu \left( \bigsqcup_{i=1}^{\infty} A_i \right) = \sum_{i=1}^{\infty} \mu(A_i) $
  \end{enumerate}
\end{definition}

% ---------------------------------------------------------------------------------------------

% \subsection{Discrete measure}

For a simpler discrete OT case we got to define a discrete measure:

\begin{definition}[Discrete measure]
A discrete measure with weights $\alpha$ and locations $x_{1}, \dots, x_{n} \in \mathcal{X}$ reads
$$
a = \sum_{i=1}^{n} \alpha_{i} \delta_{x_{i}}
$$
where $\delta_{x}$ is the Dirac delta function, which is
\[ \delta_{x_i} (A) = \begin{cases}
  1 & x \in A \\
  0 & x \not\in A
\end{cases}. \]
\end{definition}

Probability vectors gives a probability point mass in a vector form.
For each of the outcomes of the random variable corresponds one row/column in the vector.
$$
x_{0} = \begin{pmatrix}
0.25 & 0.5 & 0.1 & 0.15
\end{pmatrix}
$$

% ---------------------------------------------------------------------------------------------

\subsection{Simplex}

Let $x = (x_{0}, x_{1}, x_{2}, \dots, x_{n})$ be a probability vector
$$
\sum_{i=1}^{n} x_{i} = 1
$$

So simplex should be a set of probability vectors
$$
\Sigma_{n} := \left\{ a \in \mathbb{R}_{+}^{n} : \sum_{i=1}^{n} a_{i} = 1 \right\}
$$

\subsection{General measures}

\todo{Radon measures $\mathcal{M(X)}$ }




