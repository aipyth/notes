
% ---------------------------------------------------------------------------------------------

\subsection{Simplex}


Probability vectors gives a probability point mass in a vector form.
For each of the outcomes of the random variable corresponds one row/column in the vector.
$$
\mathbf{x} = \begin{pmatrix}
0.25 & 0.5 & 0.1 & 0.15
\end{pmatrix}
$$

Let $\mathbf{x} = (x_{0}, x_{1}, x_{2}, \dots, x_{n})$ be a probability vector
with $x_0, x_1, \ldots, x_n \ge 0$ such that
$$
\sum_{i=0}^{n} x_{i} = 1
$$

So simplex should be a set of probability vectors
$$
\Sigma_{n} := \left\{
  \mathbf{a} = \left( a_0, a_1, \ldots, a_n \right)  \in \mathbb{R}_{+}^{n} :
\sum_{i=1}^{n} a_{i} = 1,
a_i \ge 0, \;\; \forall i \in  [[n]]
\right\}
$$

\begin{definition}[Discrete measure]
  A discrete measure with weights $\mathbf{a}$ and locations $x_{1}, \dots, x_{n} \in \mathcal{X}$ reads
$$
a = \sum_{i=1}^{n} \mathbf{a}_{i} \delta_{x_{i}}
\qquad \mathbf{a}_i \ge 0 \quad \forall i \in  [[n]]
$$
where $\delta_{x}$ is the Dirac delta function, which is
\[ \delta_{x_i} (A) = \begin{cases}
  1 & x \in A \\
  0 & x \not\in A
\end{cases}. \]
\end{definition}

\paragraph{Histogram}

Let $\xi$ be a random variable
(with some continuous density function $f(x)$ which is unknown).

Let $X_1, X_2, \ldots, X_n \sim \xi$ a sample.

\begin{definition}[Histogram]
  Piecewise constant function
  \[ f_n(x) = \frac{\nu_r}{ n \cdot |\mathcal{I}_r|} \mathbbm{1} (x \in \mathcal{I}_r), \quad r \in [[m]] \]
  is called a histogram, where
  \begin{itemize}
    \item $\mathcal{I}_r$ is the division segment of the division $\mathcal{I}_1, \mathcal{I}_2, \ldots, \mathcal{I}_m$ of the area $\mathcal{I}$ of possible values of $\xi$ ;
    \item $\nu_r = \sum_{j=1}^{n} \mathbbm{1} \left( X_j \in \mathcal{I}_r \right)$ ---
      number of elements of the sample that are in $\mathcal{I}_r$.
  \end{itemize}
\end{definition}

\begin{remark}
  The histogram function for large $n$ and small enough division of the
  interval is the approximation of the true density $f(x)$.
\end{remark}
\begin{proof}
  By the Law of Large Numbers:
  $\frac{\nu_r}{n} = \frac{\sum_{j=1}^{n} \mathbbm{1} (X_j \in \mathcal{I}_r)}{n}
  \underset{n\to \infty}{\overset{P}{\longrightarrow}}
  \mathbf{E} \left[ \mathbbm{1} \left( X_1 \in \mathcal{I}_r \right) \right]
  = P\left( X_1 \in \mathcal{I}_r \right) = \int_{\mathcal{I}_r} f(x) dx$.

  We can also conclude that for some point $x_r \in  \mathcal{I}_r$
  \[ \int_{\mathcal{I}_r} f(x) dx = |\mathcal{I}_r| \cdot f(x_r) \] 
  is true because of the theorem about the mean
  and that the function $f(x)$ is continuous.
  
  Then pick $n \to \infty$ and the division infinitely small, which gives us:
  \[ \frac{\nu_r}{n \cdot |\mathcal{I}_r|} \approx f(x_r) \] 
\end{proof}


