\subsection{Kantorovich Relaxation}

$$
\mathbf{U}(\mathbf{a}, \mathbf{b}) := \left\{
\mathbf{P} \in \mathbb{R}_{+}^{n\times m} \; : \;
\mathbf{P} \mathbbm{1}_{m} = \mathbf{a}
\; \text { and } \;
\mathbf{P}^{T} \mathbbm{1}_{n} = \mathbf{b}
\right\}
$$
where $ \mathbbm{1}_{n} = \begin{pmatrix} a_{i} = 1, i = \overline{1, n} \end{pmatrix} $.

Kantorovich optimal transport reads:
$$
L_{\mathbf{C}}(\mathbf{a}, \mathbf{b}) :=
\min_{\mathbf{P} \in \mathbf{U}(\mathbf{a}, \mathbf{b})}
\left< \mathbf{C}, \mathbf{P} \right> :=
\sum_{i, j} \mathbf{C}_{i,j} \mathbf{P}_{i,j}
$$

For discrete measures of the form
\[  \] 
we store matrix $C$ as all the pairwise consts between points in the supports of $\alpha, \beta$
\[ \mathbf{C}_{i,j} \equiv c(x_{i}, y_{j}) \]
defining
\[ \mathcal{L}_{c}(\alpha, \beta) \equiv L_{\mathbf{C}} (\mathbf{a}, \mathbf{b}) \] 

That means that the this formulation of optimal transport between discrete measures is the same as the
problem between their associated probability weight vectors
$\mathbf{a}, \mathbf{b}$ except that the cost matrix $\mathbf{C}$
depends on the support of $\alpha, \beta$.
