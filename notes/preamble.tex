% Some basic packages

% \usepackage[utf8]{inputenc}
% \usepackage[T1]{fontenc}
% \usepackage{textcomp}
% \usepackage[dutch]{babel}

\usepackage[T1,T2A]{fontenc}
\usepackage[utf8]{inputenc}
% \usepackage[english,russian]{babel}

% \usepackage[utf8]{inputenc}
% \usepackage[T2A]{fontenc}

\usepackage{url}
\usepackage{graphicx}
\usepackage{float}
\usepackage{booktabs}
\usepackage{enumitem}

\usepackage{ dsfont }

\newcount\pdfminorversion
\pdfminorversion=7

% Don't indent paragraphs, leave some space between them
\usepackage{parskip}

% Hide page number when page is empty
\usepackage{emptypage}
\usepackage{subcaption}
\usepackage{multicol}
\usepackage{xcolor}

% page size settings
\usepackage{fullpage}
\usepackage[top=2cm, bottom=4.5cm, left=2.5cm, right=2.5cm]{geometry}
% \usepackage[top=2cm, bottom=1.5cm, left=4cm, right=4cm]{geometry}
% \usepackage[top=1cm, bottom=1.5cm, outer=2cm, inner=2cm, heightrounded, marginparwidth=3cm, marginparsep=1cm,headsep=0.6cm]{geometry}

% for notes at the left side
\usepackage{marginnote}

% Other font I sometimes use.
% \usepackage{cmbright}

\usepackage{hyperref}
\hypersetup{%
  colorlinks=true,
  linkcolor=blue,
  linkbordercolor={0 0 1}
}

% force display math for all environements
\everymath{\displaystyle}

\usepackage{witharrows}

% Math stuff
\usepackage{amsmath, amsfonts, mathtools, amsthm, amssymb, bbm}
% Fancy script capitals
\usepackage{mathrsfs}
\usepackage{cancel}
% Bold math
\usepackage{bm}
% Some shortcuts
\newcommand\N{\ensuremath{\mathbb{N}}}
\newcommand\R{\ensuremath{\mathbb{R}}}
\newcommand\Z{\ensuremath{\mathbb{Z}}}
\renewcommand\O{\ensuremath{\emptyset}}
\newcommand\Q{\ensuremath{\mathbb{Q}}}
% \newcommand\C{\ensuremath{\mathbb{C}}}

% Easily typeset systems of equations (French package)
% \usepackage{systeme}

% Put x \to \infty below \lim
\let\svlim\lim\def\lim{\svlim\limits}

%Make implies and impliedby shorter
\let\implies\Rightarrow
\let\impliedby\Leftarrow
\let\iff\Leftrightarrow
\let\epsilon\varepsilon

% Add \contra symbol to denote contradiction
\usepackage{stmaryrd} % for \lightning
\newcommand\contra{\scalebox{1.5}{$\lightning$}}

% \let\phi\varphi

% Command for short corrections
% Usage: 1+1=\correct{3}{2}

\definecolor{correct}{HTML}{009900}
\newcommand\correct[2]{\ensuremath{\:}{\color{red}{#1}}\ensuremath{\to }{\color{correct}{#2}}\ensuremath{\:}}
\newcommand\green[1]{{\color{correct}{#1}}}

% horizontal rule
\newcommand\hr{
    \noindent\rule[0.5ex]{\linewidth}{0.5pt}
}

% hide parts
\newcommand\hide[1]{}

% si unitx
% \usepackage{siunitx}
% \sisetup{locale = FR}

% Environments
% \makeatother
% For box around Definition, Theorem, \ldots

% \usepackage{mdframed}
% \mdfsetup{skipabove=2em,skipbelow=0em}

% \theoremstyle{definition}

% \newmdtheoremenv[nobreak=true]{definition}{Definition}
% \newmdtheoremenv[nobreak=true]{characteristic}{Characteristic}
% \newmdtheoremenv[nobreak=true]{corollary}{Corollary}
% \newmdtheoremenv[nobreak=true]{lemma}{Lemma}
% \newmdtheoremenv[nobreak=true]{proposition}{Proposition} % твердження
% \newmdtheoremenv[nobreak=true]{law}{Law}
% \newmdtheoremenv[nobreak=true]{postulate}{Postulate}
% \newtheorem*{consequence}{Consequence}
% \newtheorem*{practical}{Practical}
% \newtheorem*{terminology}{Terminology}
% \newtheorem*{example}{Example}

% \newmdtheoremenv[nobreak=true]{definition}{Definition}
% \newtheorem*{eg}{Example}
% \newtheorem*{notation}{Notation}
% \newtheorem*{previouslyseen}{As previously seen}
% \newtheorem*{remark}{Remark}
% \newtheorem*{note}{Note}
% \newtheorem*{problem}{Problem}
% \newtheorem*{observe}{Observe}
% \newtheorem*{property}{Property}
% \newtheorem*{intuition}{Intuition}
% \newmdtheoremenv[nobreak=true]{prop}{Proposition}
% \newmdtheoremenv[nobreak=true]{theorem}{Theorem}
% \newmdtheoremenv[nobreak=true]{corollary}{Corollary}

\theoremstyle{definition}
\newtheorem{definition}{Definition}[section]
\newtheorem{theorem}{Theorem}[section]
\newtheorem{corollary}{Corollary}[theorem]
\newtheorem{proposition}{Proposition}[theorem]
\newtheorem{lemma}[theorem]{Lemma}

\newtheorem{remark}[remark]{Remark}


% End example and intermezzo environments with a small diamond (just like proof
% environments end with a small square)
\usepackage{etoolbox}
\AtEndEnvironment{vb}{\null\hfill$\diamond$}%
\AtEndEnvironment{intermezzo}{\null\hfill$\diamond$}%
% \AtEndEnvironment{opmerking}{\null\hfill$\diamond$}%

% Fix some spacing
% http://tex.stackexchange.com/questions/22119/how-can-i-change-the-spacing-before-theorems-with-amsthm
\makeatletter
\def\thm@space@setup{%
  \thm@preskip=\parskip \thm@postskip=0pt
}




% These are the fancy headers
\usepackage{fancyhdr}
\pagestyle{fancy}

% LE: left even
% RO: right odd
% CE, CO: center even, center odd
% My name for when I print my lecture notes to use for an open book exam.
% \fancyhead[LE,RO]{Gilles Castel}

% \fancyhead[RO,L]{\@lecture} % Right odd,  Left even
\fancyhead[R,LO]{}          % Right even, Left odd
% \setlength{\headheight}{15pt}
\setlength{\headheight}{2pt}

\fancyfoot[RO,L]{\thepage}  % Right odd,  Left even
\fancyfoot[R,LO]{}          % Right even, Left odd
% \fancyfoot[C]{\leftmark}     % Center
\renewcommand{\headrulewidth}{0pt}

\makeatother

\DeclareMathOperator{\spn}{span}




% Todonotes and inline notes in fancy boxes
% \usepackage{todonotes}
\usepackage{tcolorbox}

\newcommand{\todo}[1]{
\textcolor{red}{ \textit{ (TODO: #1) } }
}

% Make boxes breakable
\tcbuselibrary{breakable}

\newcommand{\boxexample}[2]{
  \begin{tcolorbox}[colback=black!5!white,colframe=black,title={Example: #1}]
    #2
  \end{tcolorbox}
}

% Verbetering is correction in Dutch
% Usage: 
% \begin{verbetering}
%     Lorem ipsum dolor sit amet, consetetur sadipscing elitr, sed diam nonumy eirmod
%     tempor invidunt ut labore et dolore magna aliquyam erat, sed diam voluptua. At
%     vero eos et accusam et justo duo dolores et ea rebum. Stet clita kasd gubergren,
%     no sea takimata sanctus est Lorem ipsum dolor sit amet.
% \end{verbetering}
% \newenvironment{verbetering}{\begin{tcolorbox}[
\newenvironment{exercise}{\begin{tcolorbox}[
    arc=0mm,
    colback=white,
    colframe=green!60!black,
    title=Exercise,
    fonttitle=\sffamily,
    breakable
]}{\end{tcolorbox}}

% Noot is note in Dutch. Same as 'verbetering' but color of box is different
% \newenvironment{noot}[1]{\begin{tcolorbox}[
\newenvironment{note}[1]{\begin{tcolorbox}[
    arc=0mm,
    colback=white,
    colframe=white!60!black,
    title=Note,
    fonttitle=\sffamily,
    breakable
]}{\end{tcolorbox}}




% Figure support as explained in my blog post.
\usepackage{import}
\usepackage{xifthen}
\usepackage{pdfpages}
\usepackage{transparent}
\newcommand{\incfig}[1]{%
    \def\svgwidth{\columnwidth}
    \import{./figures/}{#1.pdf_tex}
}

% Fix some stuff
% %http://tex.stackexchange.com/questions/76273/multiple-pdfs-with-page-group-included-in-a-single-page-warning
% \pdfsuppresswarningpagegroup=1


% My name
\author{Ivan Zhytkevych}
