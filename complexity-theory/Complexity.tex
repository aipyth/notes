\documentclass[12pt,letterpaper]{report}
\usepackage{fullpage}
% \usepackage[top=2cm, bottom=4.5cm, left=2.5cm, right=2.5cm]{geometry}
\usepackage{amsmath,amsthm,amsfonts,amssymb,amscd}
\usepackage{hyperref}
\usepackage[dvipsnames]{xcolor}
\usepackage{fancyhdr}
% \usepackage{fancy}
\usepackage{mathrsfs}
\usepackage{amsmath}
\usepackage{dsfont}

\usepackage{mathtools}

\usepackage{fontspec}

\setromanfont{PTSerif}[
    Path=./fonts/,
    Extension = .ttf,
    UprightFont = *-Regular,
    BoldFont = *-Bold,
    ItalicFont = *-Italic,
    BoldItalicFont = *-BoldItalic,
]

\hypersetup{%
  colorlinks=true,
  linkcolor=blue,
  linkbordercolor={0 0 1}
}

%%%%%%%%%%%%%%%%%%%%%%%%%%%%%%%%%%%%%%%%%%%%%%%%%%%%%%%%%%%%%%%%%%%%%%%%%%%%%%
% - Define Colors
%%%%%%%%%%%%%%%%%%%%%%%%%%%%%%%%%%%%%%%%%%%%%%%%%%%%%%%%%%%%%%%%%%%%%%%%%%%%%%
\definecolor{light-gray}{gray}{0.85}

%%%%%%%%%%%%%%%%%%%%%%%%%%%%%%%%%%%%%%%%%%%%%%%%%%%%%%%%%%%%%%%%%%%%%%%%%%%%%%
\pagestyle{fancy}
\fancyhf{}
\usepackage[top=1.5cm, bottom=2cm, left=2.5cm, right=2.5cm]{geometry}
\rfoot{\rightmark}
\lfoot{\thepage}

\renewcommand{\headrulewidth}{0pt}
\renewcommand{\footrulewidth}{0.2pt}

\title{Complexity Theory Notes}
\author{}
\date{\today}

%%%%%%%%%%%%%%%%%%%%%%%%%%%%%%%%%%%%%%%%%%%%%%%%%%%%%%%%%%%%%%%%%%%%%%%%%%%%%%
% - Some declarations
%%%%%%%%%%%%%%%%%%%%%%%%%%%%%%%%%%%%%%%%%%%%%%%%%%%%%%%%%%%%%%%%%%%%%%%%%%%%%%

\newtheorem{theorem}{Theorem}
\newtheorem{claim}[theorem]{Claim}              % Вимога
\newtheorem{proposition}[theorem]{Proposition}  % Твердження
\newtheorem{lemma}[theorem]{Lemma}              % Лема
\newtheorem{corollary}[theorem]{Corollary}      % Наслідок
\newtheorem{conjecture}[theorem]{Conjecture}    % Припущення
\newtheorem*{observation}{Observation}
\newtheorem*{example}{Example}
\newtheorem*{remark}{Remark}
\newtheorem{definition}{Definition}

\DeclarePairedDelimiter{\ceil}{\lceil}{\rceil}
% \DeclarePairedDelimiter{\set}{\left\{}{\right\}}
\NewDocumentCommand{\set}{o m}{%
  % <code>
  \IfNoValueTF{#1}
    {\{#2\}}
    {\{#1 \mid #2\}}%
  % <code>
}
%%%%%%%%%%%%%%%%%%%%%%%%%%%%%%%%%%%%%%%%%%%%%%%%%%%%%%%%%%%%%%%%%%%%%%%%%%%%%%
\begin{document}
\maketitle
\tableofcontents
% \newpage
%%%%%%%%%%%%%%%%%%%%%%%%%%%%%%%%%%%%%%%%%%%%%%%%%%%%%%%%%%%%%%%%%%%%%%%%%%%%%%%
% - Contents
%%%%%%%%%%%%%%%%%%%%%%%%%%%%%%%%%%%%%%%%%%%%%%%%%%%%%%%%%%%%%%%%%%%%%%%%%%%%%%%
\chapter{Computability}

\section{Godel numeration}

\begin{enumerate}
\item $\mathcal{D}$ - subject area (objects, properly built functions, ...)
\item $\nu : \mathcal(D) \rightarrow \mathcal{N}$ - coding into unique natural number
Actually anything can be coded using this numeration.
\item there are lots of Godel numerations ($\nu(x_1, ..., x_n) = 2^{\nu(x_1)} ... p_n^{\nu(x_n)}$)
\item arithmetization of arbitrary theory (transposition from theory objects to natural numbers)
\item partial function $f : \mathbb{N}^{n} \not\to \mathbb{N}, n \in \mathbb{N}$, is (algorithmically) computable $\iff$ there is such Turing machine $M: M(x_1, ..., x_n) \simeq f(x_1, ..., x_n), \forall x_1, ..., x_n \in \mathbb{N}$ (Turing thesis)
\item set (algorithmically) computable functions coincides to set of partially\itemrecursive functions (Church thesis)
\item Turing Machines numeration $M_0, M_1, ...$
\item numeration of every (algorithmic) computable functions $\varphi_0, \varphi_1, ...$
\end{enumerate}

Example of computable function:
$$f(n) = 1, \text{ if there would be colony on Moon}$$
$$f(n) = 0, \text{ if there would be no colony on Moon}$$
subjunctive algorithm that proves computability of this function is to wait the predefined set time.

---

### Theorem about uncomputable function

Uncomputable function defined everywhere - exists.

#### Proof

\item $\varphi_0^1, \varphi_1^1, ...$ - all computable functions (арності 1)
\item $$
f(n) = \begin{cases}
    	\varphi_n^1(n) + 1, \;\; \text{ if } \; \varphi_n^1(n) \not= \bot \\
        0,  \;\;\; \text{ if } \varphi_n^1(n) = \bot \;\; (\not\exists \varphi_n^1)
\end{cases} \;,\;\;\;
n \in \mathbb{N}
$$
\item $\forall n \in \mathbb{N} f \not\simeq \varphi_n^1: f(n) \not\simeq \varphi_n^1(n)$
\item diagonalization method (Kantor)

---

### Theorem about parametrization

For arbitrary countable function $f(x, y)$ exists everywhere defined countable function $k(x)$ that $f(x, y) = \varphi_{k(x)}(y)$ for arbitrary Godel Numeration $\varphi_0, \varphi_1, ...$ unary countable functions.

#### Proof
\item $f(x, y)$ is countable $\Rightarrow \exists \text{ TM } M: M(x, y) \simeq f(x, y)$
\item $\forall a \in \mathbb{N} \exists \text{ TM } M_a : M_a(y) \simeq f(a, y)$
\item composition of Turing Machines: add argument $a$ at input (additional) tape and start TM $M$
\item number of TM $M_a$ is $k(a)$ value

#### Consequence
Number of function $k(x)$ depends only on parameter $x$.

---

### $s_n^m$ Kleene theorem (s-m-n Theorem)

#### Theorem
For arbitrary countable funcions' Godel Numeration exists such a primitiv recursive function $s: \mathbb{N}^2 \rightarrow \mathbb{N}$ (арності 2), that for arbitrary Godel number $p \in \mathbb{N}$ of some partial function of two variables next Kleene equality is true: $\varphi_{s(p, x)}(y) \simeq \varphi_p(x,y)$ for every natural number $x, y \in \mathbb{N}$. 

#### $s_n^m$ Kleene theorem (s-m-n theorem, parametrization theorem)

For arbitrary natural numbers $m,n > 0$ and arbitrary godel numeration of countable functions exists such a primitiv recursive function $s_n^m : \mathbb{N}^{m+1} \rightarrow \mathbb{N}$ that for arbitrary Godel number $p \in \mathbb{N}$ of some partial function of $m + n$ arguments Kleene equality is true:
\[ \varphi_{s_n^m(p, x_1, ..., x_m)}(y_1, ..., y_n) \simeq \varphi_p(x_1, ..., x_m, y_1, ..., y_n) \]
for every natural number $x_1, ..., x_m, y_1, ..., y_n \in \mathbb{N}$.

---

### Universal function

For arbitrary set of partial functions $\mathcal{H} \subseteq \mathcal{F}_n$ of $n$ variables, $n \in \mathbb{N}_0$, function $f \in \mathcal{F}_{n+1}$ of variables $n+1$ is called universal function of set of functions $\mathcal{H}$, if the following two conditions are true:
\item for arbitrary number $c \in \mathbb{N}_0$ function $f(c, \cdot)$ of variables $n$ is in set of functions $\mathcal{H}$
\item for arbitrary function $h$ of set of functions $\mathcal{H}$ exists such a number $c \in \mathbb{N}_0$, that $h(x_1, ..., x_n) \simeq f(c, x_1, ..., x_n)$ for arbitrary values $x_1, ..., x_n \in \mathbb{N}_0$.

Algorithm to compute universal function for a set is called universal.

#### Theorem (about numeration)

For arbitrary number $n \in \mathbb{N}_0$ exists such universal function of set of all partial computable functions $\mathcal{H}_n \subseteq \mathcal{F}_n$ of $n$ variables.

##### Proof
\item let $f(y, x_1, ..., x_n) \simeq \varphi_y(x_1, ..., x_n)$ for arbitrary numbers $y, x_1, ..., x_n \in \mathbb{N}_0$
\item by value $y \in \mathbb{N}_0$ find algorithm of computing function $\varphi_y$ and compute value $\varphi_y(x_1, ..., x_n)$ using this algorithm
\item $\Rightarrow$ function $f$ is computable

#### Theorem

For arbitrary number $n \in \mathbb{N}_0$ there is no such universal funtion of set of defined everywhere computable functions $\mathcal{H}_n^{tot} \subseteq \mathcal{F}_n^{tot} \subset \mathcal{F}_n$ of $n$ variables.

##### Proof
\item let universal function $f$ of set of functions $\mathcal{H}_n^{tot} : f(y, x_1, ..., x_n) = \varphi_y(x_1, ..., x_n)$ for arbitrary numbers $y, x_1, ..., x_n \in \mathbb{N}_0$
\item let $h(x_1, ..., x_n) = f(x_1, x_1, ..., x_n) + 1$ for arbitrary numbers $x_1, ..., x_n \in \mathbb{N}_0$
\item $\Rightarrow h \in \mathcal{H}_n^{tot} \Rightarrow \exists c \in \mathbb{N}_0 f(c, x_1, ..., x_n) = h(x_1, ..., x_n)$ for arbitrary numbers $x_1, ..., x_n \in \mathbb{N}_0$
\item on one side $h(c, ..., c) = f(c, ..., c)$ but $h(c, ..., c) = f(c, ..., c) + 1$ by definition of $h$ $\Rightarrow$ *contradiction*.

---

* every universal function of unary computable functions set defines numeration $f(x, y) = \varphi_x(y)$
* binary function $U$ is called **main universal function (main numeration)** if for any binary computable function $h$ exists such a defined everywhere computable unary funtion $g$ that $h(x, y) = U(g(x), y)$ for any numbers $x, y \in \mathbb{N}_0$
* $\Rightarrow$ exists main universal function of set of all unary computable functions
* $\Rightarrow U_1(x, y) = U_2(c_1(x), y)$ and $U_2(x, y) = U_1(c_2(x), y)$ (**theorem about main numerations' ismorphism**)
* operations on computable functions $\Leftrightarrow$ operations on their indexes

---

### Theorem about motionless point

For arbitrary Godel numeration $\varphi_0. \varphi_1, ...$ of unary computable functions and arbitrary unary computable defined everywhere funtion $f$ exists such natural number $n \in \mathbb{N}_0$ that $\varphi_n \simeq \varphi_{f(n)}$.

##### Proof
\item consider such function $\varphi_{f(\varphi_x(x))}(y)$, $\varphi_{f(\varphi_x(x))}(y) \simeq \psi(f(\varphi_x(x)), y) \simeq g(x, y) , \forall x, y \in \mathbb{N}_0$
\item from $s_n^m$ Kleene theorem follows that exists such unary defined everywhere function $h$ that $\varphi_{f(\varphi_x(x))}(y) \simeq \varphi_{h(x)}(y), \forall x, y \in \mathbb{N}_0$
\item let $h \simeq \varphi_m \Rightarrow \varphi_{f(\varphi_x(x))}(y) \simeq \varphi_{\varphi_m}(y), \forall x, y \in \mathbb{N}_0$
\item let $\varphi_m(m) = n$ (defined everywhere) $\Rightarrow \varphi_{f(n)}(y) \simeq \varphi_n(y), \forall y \in \mathbb{N}_0$

#### Second theorem about recursion (Kleene, 1938)
For arbitrary Godel numeration $\varphi_0. \varphi_1, ...$ unary omputable functions and arbitrary binary partial computable function $f$ exists such natural number $n \in \mathbb{N}_0$ that $\varphi_n(y) \simeq f(n ,y)$ for all numbers $y \in \mathbb{n}_0$.

##### Consequence
Let  function $h$ - $f(x, y) \simeq \varphi_{h(x)}(y)$ ($s_n^m$ theorem).
Let number $m$ be motionless point of function $h$.
From theorem about Rodger's motionless point follows second theorem about recursion ($n=m, \; \varphi_m(y) \simeq \varphi_{h(m)}(y) \simeq f(m,y)$ for all numbers $y \in \mathbb{N}_0$)

##### Consequence
Let function $f$ - for arbitrary algorithm $\mathcal{A}_x$ algorithm $\mathcal{A}_{f(x)}$ "prints description" of algorithm $\mathcal{A}_x$.
Function $f$ is computable $\Rightarrow$ by theorem about motionless point exists algorithm $\mathcal{A}$, that "prints own description".

### Computable functions

> Can UTM compute arbitrary function $\{0, 1\}^* \rightarrow \{0,1\}^*$?

#### Theorem
Exists uncountable function UC: $\{0,1\}^* \rightarrow \{0,1\}$

##### Proof
\item TM numeration using set $\{0,1\}^*$, for arbitrary word $x \in \{0,1\}^*$ appropriate Turing Machine is marked $M_x$ or $M_{\lceil x \rceil}$
\item define $$UC(x) = \begin{cases}
	0, \text{ if } M_x(x) = 1 \\
	1, otherwise
\end{cases} \forall x \in \{0,1\}^*
$$
\item let $\exists \text{ TM } \widetilde{M}: \forall x \in \{0,1\}^* \widetilde{M}(x) = UC(x)$
$\widetilde{M}(\lfloor\widetilde{M}\rfloor) = ?$
\item if $\widetilde{M}(\lfloor \widetilde{M} \rfloor) = 1$, then $UC(\lfloor \widetilde{M} \rfloor)=0$ and vice versa

### Modification of TM for recognition tasks

Recognition task $\Leftrightarrow$ defined everywhere function $\{0,1\}^* \rightarrow \{0,1\}$

#### Definition (multitape Turing Machine)
\item $k \in \mathbb{N}^+$ number of tapes
\item $\Gamma$ Turing Machine alphabet
\item $\# \in \Gamma$
\item $\{0,1\}^*$ input alphabet
\item $Q$ nonempty finite set of internal states
\item $q_0 \in Q$ initial state
\item $q_{acc} \in Q$ final state, that accepts input word
\item $q_{rej} \in Q, q_{acc} \not= q_{rej}$ final state that rejects input word
\item $\delta : (Q \backslash \{q_{acc}, q_{rej}\}) \times \Gamma^k \nrightarrow Q \times \Gamma^{k-1} \times \{L, S, R\}^k$ partial function of transitions

#### Notion
$q_{acc}, q_{rej} \in Q, q_{acc} \not= q_{rej}$, other ending configurations does not exist
($\Sigma = \{0,1\}, q_{acc} \equiv q_{accept} \equiv q_y \equiv q_{yes}, q_{rej} \equiv q_{reject} \equiv q_{n} \equiv q_{no}$)

#### Definition
Final configuration of TM is called positive (negative) if it's state is final state that accepts (rejects) input word.

#### Definition
TM $M$ input word $x$
\item accepts if $M(x) = 1$ ($q_{acc}$, positive configuration)
\item rejects if $M(x) = 0$
\item not  accepts if $M(x) = 0$ or $M(x) = \bot$
\item not rejects if $M(x) = 1$ or $M(x) = \bot$

### Language recognition

#### Definition
TM $M$ resolves (decides) language $L \subseteq \{0,1\}^*$
\item if $x \in L$ then $M(x) = 1$
\item if $x \not\in L$ then $M(x) = 0$

#### Definition
TM $M$ recognizes language $L \subseteq \{0,1\}^*$
\item if $x \in L$ then $M(x) = 1$
\item if $x \not\in L$ then $M(x) = 0$ or $M(x) = \bot$

#### Definition
Languages - decidable (recursive) or semidecidable (recursively countable)

language $L(M) \;\; (L_M)$ of TM $M$ - all word it accepts.

#### Definition
Turing machines $M_1$ and $M_2$ are:
\item same if there exists such permutation of inner states and/or change of directions 'left' and 'right', otherwise - is principle different
\item equivalent if $M_1 = M_2$, $M_1 \simeq M_2$
\item with one language if $L(M_1) = L(M_2)$

### **HALT** problem

Define by binary representation of TM $M$ and input word $x \in \{0,1\}^*$, will TM $M$ stop on input word $x$. (decide language $L_{HALT}$)

#### Theorem
**HALT** task is unsolvable.

##### Proof
\item let existance of $M_{HALT}$
\item $M_{diag}(x) = M_{HALT}(x, x)$
\item $$M^{co}(x) = \begin{cases}
\text{cycle}, M_{diag}(x) = 1 \\
\text{stop}, M_{diag}(x) = 0
\end{cases}$$
\item $M^{co}(\lfloor M^{co} \rfloor)$ ?

### $HALT_\varepsilon$ problem

Define by binary representation of Turing Machine whether TM $M$ will stop on empty input word (decide language $L_{HALT_\varepsilon}$).

#### Theorem

Problem $HALT_\varepsilon$ is unsolvable.

##### Proof
\item for arbitrary pair of TM $\widetilde{M}$ and input word $x$ there exists TM $\widetilde{M}_x$
\item if such TM exists, that solves $HALT_\varepsilon$ problem, then it solves $HALT$ problem
\item **contradiction**

### Rice's theorem

Numeric set $S \subseteq \mathbb{N}$ is called **invariant**, if representation of any two equivalent TM simultaneously is in or not in set $S$.

#### Examples
\item all TM, that accepts input word $11$
\item all TM, that accepts at least one input word
\item all TM, that never get hung up
\item all TM, that stop after 15 tacts with input word $1$

---
---

## Lection *

\item Tibour Rado 1962 year
\item TM model: $(1, \{0, 1\}, {1}, 0, Q, q_h, q_0, \sigma)$
\item $\sigma : (Q \backslash \{q_h\}) \times \{0,1\} \rightarrow Q \times \{0,1\} \times \{L, R\}$
\item $\mathcal{K}_{BB}$ -- all turing machines that stop on empty word (Rado class)
\item $\mathcal{K}_{BB}(n)$ -- all TM that stop on empty imput word and have $n$ non-final states
\item $s(M)$ -- number of steps that TM M does with empty input word before it stops
\item $\sigma(M)$ -- number of non-empty left cells on tape after TM M stops with empty input word

#### Definition 
Rado functions -- $S, \Sigma : \mathbb{N} \rightarrow \mathbb{N}$ that for arbitrary natural number $n \in \mathbb{N}$ take value $S(n) = \max_{M \in \mathcal{K}_{BB}(n)} s(M)$ and $\Sigma(n) = \max_{M \in \mathcal{K}_{BB}(n)} \sigma(M)$

##### Consequence
For arbitrary natural number $n \in \mathbb{N}$ the following is true $\Sigma(n) \leq S(n)$.

##### Statement
For arbitrary natural number $n \in \mathbb{N}$ class Rado $\mathcal{K}_{BB}$ cardinality is limited at top $4^{n + 1}^{2n}$

### Theorem 
For arbitrary computable function $f : \mathbb{N} \rightarrow \mathbb{N}$ exists sucvh natural number $n_f \in \mathbb{N}$, that $\Sigma(n) > f(n)$ for all natural numbers $n \in \mathbb{N}, n > n_f$.

#### Consequence
\item $\Sigma$ is uncomputable
\item $S$ is uncomputable

\item for arbitrary computance and arithmetically right theory exists such natural number $k \in \mathbb{N}$ that for arbitrary number $n \in \mathbb{N}, n \geq k$ no statement like $S(n) = m$, where $m \in \mathbb{N}$ cannot be proved in cases of the theory
\item in cases of the Cermello-fRenkel theory cannot be computed $S(748)$
\item built TM from Rado class $\mathcal{K}_{BB}(1919)$ that stops when CF theory with axiom of choice is contradictional
\item - built TM from Rado class $\mathcal{K}_{BB}(744)$ that stops that Rieman hypothesis is wrong

\item small by building TM may generate big computance
\item exists some limit of computable functions, and quite rapid by grows functions are uncomputable

## Lection: Reducibility

#### Definition
Reducibility is some procedure (or algorithm) of translating one problem to another.
Problem P1 reduces to problem P2:
\item use available solution of P2
\item prove complexity of solution of problem P1


#### Example
\item P1 : $ax^2 + bx + c = 0$
\item P2 : $x^2 + bx + 1 = 0$
\item P3 : $x^2 -2x+1=0$
\item P3 reduces to P2
\item P2 reduces to P1

\item PRIME — for given natural number define whether it is prime
\item COMPOSITE — for given natural number define whether it has non-trivial dividers (not 1 and itself)
\item FACTOR — for given natural numbers $m$ and $k$ define whether number $m$ has non-trivial divider that is $\not\geq k$
\item COMPOSITE -> PRIME
\item задача PRIME -> COMPOSITE
\item PRIME & COMPOSITE -> FACTOR
\item FACTOR not -> PRIME | COMPOSITE

#### Definition
Reduction of languages is arbitrary binary relation on set $\{0,1\}^*$ that is reflexive and transitional. Laguage $L_1 \subseteq \{0,1\}$ reduces to language $L_2 \subseteq \{0,1\}$ if ordered pair $(L_1,L_2)$ belong to the binary relation that defines reduction. For language reduction signing $\leq$ is used with probable use of under- and upper- indexes for clarification of specifical reduction.

##### Consequence
Computable problem P1 reduces to computable problem P2 if such encoding schemas $e_1, e_2$ of P1 and P2 exists, that language $L[P1, e_1]$ reduces to $L[P2, e_2]$.

##### Remark
Is applicable for arbitrary sets.

\end{document}

