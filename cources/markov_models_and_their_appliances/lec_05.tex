\lecture{5}{Tue 27 Sep 2022 12:20}{Ланцюги Маркова. Означення та основні властивості.}

Нехай $\left\{ X_n \right\}_{n \geq 0} $ --- послідовність в.в. заданих на не більш ніж зліченній
множині $E = \{1, 2, \ldots\} $.

\begin{definition}
  $\left\{ X_n \right\}_{n \geq 0} $ утворює ланцюг маркова з початковим розподілом
  $\lambda = \left( \lambda_i \right)_{i \in  E}$ та матрицю похідних ймовірностей
  $P = \left( p_{ij} \right)_{i,j \in  E} $ якщо:
  \begin{enumerate}
    \item $P\left( X_0 = i \right) = \lambda_i, \; i \in  E$
  \item $P\left( X_{n+1} = j | X_n = i_n, \ldots, X_0 = i_0 \right) =
    P\left( X_{n+1 = j} \mid X_n = i_n\right) = p_{i_n,j} $
  \end{enumerate}
  \[ \forall n \geq 0 \text{ та } \forall i_0, i_1, \ldots, i_n , j \in E \] 
\end{definition}

\begin{remark}
  Розглядаємо однорідні ланцюги Маркова, для яких перехідні ймовірності
  \[ P\left( X_{n+1} = j | X_n = i \right) = p_{ij} \] 
  не залежить від $n$.

  Це означає, що $P\left( X_1 = j, X_0 = i \right) = P\left( X_2 = j, X_1 = i \right) = \ldots $
\end{remark}

Матриця $P = \left( p_{ij} \right)_{i,j \in  E} $ є стохастичною
\[ p_{ij} \geq 0 \quad \text{ та } \quad \sum_{j \in  E}^{} p_{ij} = 1 \] 


\begin{lemma}
  Нехай $E$ та $F$ --- дві зліченні (скінченні) множини і
  нехай
   \[ f: \N \times E \times F \to  E \] 
  Нехай $X_0, Y_1, \ldots$ --- незалежні в сукупності в.в. такі що
  $X_0$ прйимає значення в $E$;  $Y_n$ приймають значення в  $F$.
  Нехай  $\{X_n, n\geq 1\}$ означено наступним чином:
  \[ X_{n+1} = f\left( n ,X_n, Y_{n+1} \right), \quad n \in \N  \] 
  тоді $\{X_n\} _{n\geq 0}$ --- ланцюг Маркова.
\end{lemma}

\textit{Написати доведення}

\begin{example}[Підкидання грального кубика]
  $X_n$ --- кількість шісток, які випали до  $n$-го підкидання. 
\[ X_{n+1}  \begin{cases}
  X_n & \text{ якщо в $\left( n+1 \right)$-му підкиданні не випала шістка} \\ 
  X_n + 1 & \text{ якщо шістка випала у $(n+1)$-му підкиданні }
\end{cases} \] 

Тоді $X_{n+1} = X_n + Y_{n+1}$, де $\{Y_n\}_{n \geq 0}$ --- н.о.р. в.в. з 
\[ P\left( Y_n = 0 \right) =\frac{5}{6} ; \quad P\left( Y_n = 1 \right) = \frac{1}{6} \] 

$\left\{ X_n \right\}_{n \geq 0} $ --- ланцюг Маркова.

Множина станів $E = \{0,1,2,\ldots\} $.

Перехідні ймовірності:
\[ p_{ij}=P\left( X_{n+1} = j | X_n = i \right) = \begin{cases}
  5/6 & j = i \\
  1/6 & j = i+1 \\
  0 & \text{інакше}
\end{cases} \] 
\end{example}

\begin{example}[Гіллястий процес Ватсона-Гальтона]
  Нехай спостерігається деяка популяція частинок, еволюція яких така: кожна
  частинка незалежно від інших в деякі моменти (дискретні) часу $n = 0, 1, 2, \ldots$
  перетворюється на $i$ частинок з ймовірністю $p_i: \sum_{i=0}^{\infty} p_i = 1$.
  
  Стан системи: $X_n$ --- кількість частинок в момент часу $n$ (в  $n$-му поколінні).

   \[ X_{n+1} = Y^{(1)} + Y^{(2)} + \ldots + Y^{(X_n)} \]
   де $Y^{(i)}$ --- кількість нащадків $i$-ї частинки з попереднього покоління.
\end{example}

\begin{example}[Модель Еренфестів дифузії молекул газу]
  Є дві урни $A$ та $B$, в яких знаходяться  $N$ частинок. В кожен момент часу
  навмання вибирається число з $\{1, \ldots, N\}$ і частинка з відповідним номером
  перекладається в іншу урну.

  $X_n$ --- кількість частинок в $A$ в момент $n$.
  \[ X_{n+1} = \begin{cases}
    X_{n} - 1 & A \to  B \\
    X_n + 1, B \to  A
  \end{cases} \] 

  Припустимо, що $X_n = i$.

  \[ p_{i,i-1} = \frac{i}{N}; \; p_{i,i+1} = \frac{N-1}{N}; \; p_{ij} = 0 \text{ при } j \neq i \pm 1 \] 

  \[ P = \begin{bmatrix} 
    0 & 1 & 0 & \ldots && 0 \\
    \frac{1}{N} & 0 & 1 - \frac{1}{N} & \ldots && 0 \\
    0 & \frac{2}{N} & 0 & 1 - \frac{2}{N} & \ldots & 0 \\
    \vdots & \vdots & \ddots & \ddots & \ddots & \vdots \\
    \ldots & \ldots & \ldots & \ldots & 1 & 0
  \end{bmatrix}  \] 

  \[ P(i \to i+1) > P(i\to i-1) \text{ при } i < \frac{N}{2} \] 
  \[ P(i\to i+1) < P(i\to i-1) \text{ при } i > \frac{N}{2} \] 
\end{example}


\subsection{Властивості ланцюга Маркова}
\subsubsection{Скінченновимірні розподіли}

\begin{enumerate}
  \item 

\begin{align*}
   P\left( X_0 = i_0, x_1=i_1, \ldots, X_n = i_n \right) = \\
   = P\left( X_0 = i_0 \right) \cdot P(X_1 = i_1 | X_0 = i_0) \cdot  P\left( X_2 = i_2 | X_1 = i_1, X_0 = i_0 \right) \times \\
   \times P\left( X_n = i_n | X_{n-1} = i_{n-1}, \ldots, X_0 = i_0 \right) = \\
   = P(X_0 = i_0) P\left( X_1 = i_1 | X_0 = i_0 \right) \cdot \ldots\cdot P\left( 
   X_n = i_n | X_{n-1} = i_{n-1}\right) =
   \boxed{\lambda_{i_0} \cdot  p_{i_0 i_1} \cdot \ldots \cdot p_{i_{n-1} i_n}}
.\end{align*}

\item
\[ P\left( X_n = j | X_0 = i \right) = \frac{P\left( X_n = j, X_0 = i \right) }{P(X_0 = i)} \] 

\begin{align*}
  P\left( X_n = j, X_0 = i \right) = \sum_{i_1, i_2, \ldots, i_{n-1} \in  E}^{} P\left( X_n = j, X_{n-1} = i_{n-1}, \ldots,
  X_1 = i_1, X_0 = i \right) = \\
  \sum_{i_1, \ldots, i_{n-1} \in  E}^{} \lambda_i \cdot  p_{ii_1} \cdot  p_{i_1 i_2} \cdot * p_{i_{n-1}j} =
  \boxed{\lambda_i \cdot  \left( P^{n} \right) _{ij}}
.\end{align*}

Таким чином:
\[ P\left( X_n = j | X_0 = i \right) = \frac{\lambda _i \cdot \left( P^{n}_{ij} \right) }{\lambda _i} = \left( P^{n}_{ij} \right)  \] 

Отже \[ \boxed{ p^{(n)}_{ij} = P\left( X_n = j |X_0 = i \right) = \left( P^{n} \right) _{ij} } \] 

\item
\begin{align*}
  P\left( X_n = j \right) = \sum_{i \in E}^{} P\left( X_0 = i \right) \cdot  P\left( X_n = j | X_0 = i \right) = \\
  \sum_{i \in  E}^{} \lambda _i \cdot \left( P^{n} \right) _{ij}
.\end{align*}

Оскільки $\left( P^{n} \right)_{ij} = P\left( X_n = j| X_0 = i \right) $ 
то \[ \sum_{j \in  E}^{} \left( P^{n} \right) _{ij} = \sum_{j \in  E}^{} P\left( X_n = j |X_0 = i \right) = 1 \]

Отже $P^{n}$ --- стохастична для довільного $n$.

\end{enumerate}

\begin{theorem}[Марковська властивість]
  Нехай $\left\{ X_n \right\}_{n \geq 0} $ --- ланцюг Маркова $(\lambda, P)$.

  Тоді \[ \forall M \geq 1 \text{ та } \forall i \in  E \] при умові, що $X_m = i$
  послідовність  $\left\{ X_{m+n}, n \geq 0 \right\}$ є ланцюгом Маркова з початковим розподілом
  \[ \delta_i = \left( 0 \ldots 0, 1, 0 \ldots 0 \right) \] 

  і матрицею перехідних ймовірностей $P$.

  Більше того: розподіл в.в. $X_{m+1}, X_{m+2}, \ldots$ при $X_{m}=1$ не залежить від
  $X_0, \ldots, X_{m-1}$.
\end{theorem}

\begin{proof}
  Достатньо довести, що \[ 
   \] 
  \begin{align*}
    P &\left( X_{m+1} = j_1, \ldots, X_{m+k} = j_k | X_{m} =i, X_{m-1} = i_{m-1}, \ldots, X_0 = i_0 \right) =\\
      &= P\left( X_1 = j_1, \ldots, X_k = j_k | X_0 = i \right) 
  .\end{align*}

  \begin{align*}
    P\left( X_{m+1} = j_1, \ldots, X_{m+k} = j_k | X_{m} = i ,\ldots,X_0 = i_0 \right) = \\
    = \frac{P\left( X_0 = i_0, \ldots, X_{m-1} = i_{m-1}, X_{m}=i, \ldots, X_{m+k} = j_k \right) }{
    P\left( X_0 = i_0, \ldots, X_m = i \right) } = \\
    =  \frac{\left[ \lambda_0 p_{i_0 i_1} \cdot  \ldots \cdot p_{i_{m-1} i} \right] \cdot p_{ij_1} \ldots p_{j_{k-1}j_k}}{\lambda _0 p_{i_0 i_1} \ldots p_{i_{m-1} i}} = \\
= p_{i j_1} \cdot  \ldots \cdot p_{j_{k-1} j_k} = P\left( Y_1 = j_1, \ldots, Y_k = j_k | Y_0 = i \right) 
  .\end{align*}
  де $\left\{ Y_k \right\}_{k \geq 0}$ --- ланцюг Маркова $\left( \delta, P \right)$, де
  $\delta = \left( 0 \ldots 0, 1, 0 \ldots 0 \right) $.
\end{proof}

\begin{example}
  \[ E = \{1, 2\} ; \quad P = \begin{pmatrix}
    \frac{1}{2} & \frac{1}{2} \\
    \frac{1}{3} & \frac{2}{3}
  \end{pmatrix} , \quad \lambda = \left( \frac{1}{2}, \frac{1}{2} \right)  \] 
  \begin{figure}[ht]
    \centering
    \incfig{example-states-transition}
    \caption{Example: states transition}
    \label{fig:example-states-transition}
  \end{figure}

  \begin{align*}
    P\left( X_0 = 1, X_1 = 2 \right) = P\left( X_0 = 1 \right) \cdot 
    P\left( X_1 = 2 | X_0 = 1 \right) = \\
    = \lambda _1 \cdot  p_{12} = \frac{1}{2} \cdot  \frac{1}{2} = \frac{1}{4}
  .\end{align*}

  \begin{align*}
    P\left( X_3 = 1 | X_1 = 2 \right) = p^{(2)}_{21} = \left( p^{2} \right) _{21} \\
    p^2 = \begin{pmatrix} \frac{1}{2} & \frac{1}{2} \\ \frac{1}{3} & \frac{2}{3} \end{pmatrix} \cdot \begin{pmatrix} 
  \frac{1}{2} & \frac{1}{2} \\ \frac{1}{3} & \frac{2}{3} \end{pmatrix} = 
  \begin{pmatrix} \cdot & \cdot \\ \boxed{\frac{7}{18}} & \cdot \end{pmatrix}
  .\end{align*}

  \begin{align*}
    P\left( X_3 = 1 \right) = \sum_{i \in E}^{} P\left( X_0 = i \right) \cdot 
    P\left( X_3 = 1 | X_0 = i \right) = \frac{1}{2} * p^{(3)}_{11} + \frac{1}{2}
    p^{(3)}_{21} = \frac{1}{2}\left( (p^{(3)})_{11} + (p^{(3)})_{21} \right) 
  .\end{align*}
\end{example}


\subsection{Класифікація станів ланцюга Маркова}

\begin{definition}
  Стан $i$ назхивається несуттєвим якщо
  \[ \exists j \in  E \;\; (j \neq i) : \;
    p^{(m)}_{ij} > 0 \; \text{ при } m \geq 1 
  \text{ але } p^{(n)} _{ji} = 0 \;\; \forall n \geq 1\] 
\end{definition}

\begin{example}
  Ланцюг, в якого всі стани є несуттєвими.
  \begin{align*}
    E &= \{1, 2, 3, \ldots \} \\
    p_{ij} &= \begin{cases}
      p & j = i \\
      1 - p & j = i+1 \\
      0 & \text{ інакше }
    \end{cases}
  .\end{align*}
\end{example}

\begin{remark}
  Якщо $E$ --- скінченна, то всі стани не можуть бути несуттєвими.
\end{remark}

Надалі вважаємо, що $E$ --- множина суттєвих станів.

\begin{definition}
  % \begin{enumerate}
  %   \item 
  % Стан $j \in  E$ наз. \textbf{досяжним} зі стану $i \in  E \; \left( i \to  j \right) $
  % якщо $\exists m > 0: p^{(m)}_{ij} > 0$.
  % \item
  % Стан $i, j$ називається \textbf{сполучним} $(i \to  j)$, якщо
  % $\exists m_1,m_2 > 0: p^{(m_1)}_{ij} > 0, \; p^{(m_2)}_{ji} > 0$.
  % \end{enumerate}
  Стан $j \in  E$ наз. \textbf{досяжним} зі стану $i \in  E \; \left( i \to  j \right) $
  якщо $\exists m > 0: p^{(m)}_{ij} > 0$.

  Стан $i, j$ називається \textbf{сполучним} $(i \leftrightarrow  j)$, якщо
  $\exists m_1,m_2 > 0: p^{(m_1)}_{ij} > 0, \; p^{(m_2)}_{ji} > 0$.
\end{definition}

\begin{remark}
  Відношення $\leftrightarrow$ є відношенням еквівалетності: симметричне, рефлексивне та транзитивне.
\end{remark}
\begin{proof}
  Справді: якщо $i \leftrightarrow j, j \leftrightarrow i $, то
   \[ \exists m_1, m_2 > 0 : \quad p^{(m_1)}_{ij} > 0, \;\; p^{(m_2)}_{ji} > 0 \] 
   \[ \exists r_1, r_2 > 0 : \quad p^{(r_1)}_{jk} > 0, \;\; p^{(r_2)}_{kj} > 0 \] 
\end{proof}

Тоді
\[  \] 
\begin{align*}
  p^{(m_1 + r_1)}_{ik} &= \sum_{l \in  E}^{} P\left( X_{m_1 + r_1} = k, X_{m_1} = l | X_0 = i \right) = \\
  &= \sum_{l \in  E}^{} P\left( X_{m_1} = l | X_0 = i \right) \cdot 
  P\left( X_{m_1 + r_1} = k | X_{m_1} = l, X_0 = i \right) = \\
  &= \sum_{l \in E}^{} p^{(m_1)}_{il} \cdot p^{(r_1)}_{lk} 
  \geq p^{(m_1)}_{ij} p^{(r_1)}_{jk} > 0
.\end{align*}

Аналогічно
\[ p^{(m_2 + r_2)}_{ki} = \sum_{l \in E}^{} p^{(m_2)}_{kl}p^{(r_2)}_{li} 
\geq p^{(m_2)}_{kj} p^{(r_2)}_{ji} > 0 \] 

Тоді $E = \bigsqcup\limits^{S}_{i=1} E_i $, де $E_1, \ldots, E_S$ --- неперетинні
класи сполучних станів.

\begin{definition}
  Ланцюг, всі стани якого утворюють один клас сполучних станів,
  називається нерозкладним.
\end{definition}

\begin{example}
  \[ E = \{1, 2, 3, 4\} \] 
  \[ P = \begin{pmatrix}
    0 & 0.5 & 0.5 & 0 \\
    0.5 & 0 & 0 & 0.5 \\
    0 & 0 & 0.5 & 0.5 \\
    0 & 0 & 0.5 & 0.5
  \end{pmatrix}  \] 

\begin{figure}[ht]
    \centering
    \incfig{приклад-нерозкладного-ланцюга-маркова}
    \caption{Приклад нерозкладного ланцюга Маркова}
    \label{fig:приклад-нерозкладного-ланцюга-маркова}
\end{figure}
\end{example}

Охарактеризуємо стани:
\[ 1 \to 3 \text{ але } 3 \not\to 1 \implies \text{ 1 несуттєвий }\] 
\[ 2 \to 4 \text{ але } 4 \not\to 2 \implies \text{ 2 несуттєвий } \] 

3, 4 --- суттєві стани. $\widetilde{E} = \{3,4\}$ --- клас сполучних станів.

\begin{definition}
  Стан $i \in  E$ має період $d = d(i)$, якщо:
  \begin{enumerate}
    \item $p^{(n)}_{ii} > 0$ лише для тих $n $, які кратні $d$
    \item  $d$ --- найбільше з чисел, які володіють властивістю 1.
  \end{enumerate}
  \[ d = GCD \{ n : p^{(n)}_{ii} > 0 \} \] 
\end{definition}

\begin{definition}
  Якщо $d = 1$, то стан називається \textbf{аперіодичним}.
\end{definition}

\begin{definition}
  Якщо \textbf{всі стани} нерозкладного ланцюга Маркова є аперіодичними, то сам
  \textbf{ланцюг} називається \textbf{аперіодичним}.
\end{definition}










