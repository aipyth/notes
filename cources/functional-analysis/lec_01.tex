$C[a, b]$ - set of all continuous functions

\section{Metric spaces}

\begin{definition}[Metric space]
    $X$ set. Function $d: x\times X \to [0, \infty)$ is called a metric if the following condition are met:
    \begin{enumerate}
        \item $d(x, y) = 0 \iff x = y$
        \item $d(x, y) = d(y, x)$
        \item $d(x,z) \leq d(x,y) + d(y, z)$ - the inequality of the triangle
    \end{enumerate}
\end{definition}
\begin{definition}
    $(X, d)$ - metric space.
\end{definition}
\begin{example}[Discrete space]
    $X$ - arbitrary.
    \[
    d(x, y) = \begin{cases}
            1 & x \not= y \\
            0 & x = y
        \end{cases}
    .\] 
\end{example}
\begin{example}[Real line]
    $X = \mathbb{R}, \;\; d(x, y) = \left| x - y \right| $
\end{example}
\begin{example}[n-dimentional space]
    $X = \mathbb{R}^{n} = \{x = \left( x_1, \ldots, x_n \right) \mid x_i \in \mathbb{R}, 1 \leq i \leq n \} $
    \[
    d(x, y) = \sqrt{\sum_{i=1}^{n} \left( x_i - y_i \right)^2 } 
    .\] 
\end{example}

\begin{example}
    $d_1(x, y) = \sum_{i=1}^{n} \left| x_i - y_i \right|$ - metric on $\mathbb{R}^{n}$

    \begin{proof}
    \[
    d_1(x, z) = \sum_{i=1}^{n} \left| x_i - z_i \right| \leq \sum_{i=1}^{n} \left( \left| x_i - _i \right| + \left| y_i - z_i \right|  \right) =
    d_1(x, y) + d_(y, z)
    .\] 
    \end{proof}

    $d_{\infty} (x, y) = \underset{1\leq i \leq n}{\max} \left| x_i - y_i \right|$ - metric on $\mathbb{R}^{n}$
    
    \begin{proof}
        \begin{align*}
        d_{\infty}(x, y) = 0 \iff \forall i : x_i = y_i \iff x = y \\ 
        d_{\infty}(x, z) = \underset{1\leq i \leq n}{\max} \left| x_i - z_i \right|  \leq d_{\infty}(x, y) + d_{\infty}(y, z) \\
        \left| x_i - z_i \right| \leq \left| x_i - y_i \right| + \left| y_i - z_i \right|        
        .\end{align*}
    \end{proof}
    \[
    1 \leq p < \infty : d_{p} (x, y) = \left( \sum_{i=1}^{n} \left| x_i - y_i \right| ^{p} \right) ^{\frac{1}{p}}
    .\] 
    \[
    0 < p < 1 : d_{p}(x, y) = \sum_{i=1}^{n} \left| x_i - y_i \right| ^{p}
    .\] 

\end{example}

\begin{example}
$C[a, b]$ set of all continuous functions.
$f: [a, b] \to \mathbb{R}$
\[
d(f, g) = \underset{a \leq t \leq b}{\sup} \left| f(t) - g(t) \right| 
.\] 
$d(f, g)$ is a metric on $C[a, b]$.
\end{example}

\begin{example}
    $C_b[\mathbb{R}]$ - a set of all continuous and limited functions $f: \mathbb{R} \to \mathbb{R}$.
    \[
    d(f, g) = \underset{t \in \mathbb{R}}{\sup} \left| f(t) - g(t) \right| 
    .\] 
\end{example}
\begin{example}
    $(X, d)$ - metric space, $Y \subset X$
    \[
    d(y_1, y_2), y_1, y_2 \in Y
    .\] 
    $(Y, d)$ - subspace $X$.
\end{example}

\begin{definition}
    $\left( X, d \right) $ - metric space, $\left\{ x_n : n\geq 1 \right\} $
    series of $X$ elements. $\left\{ x_n : n\geq 1 \right\} $ converges to $x \in X$ if $\underset{n\to\infty}{\lim} d(x_n, x) = 0$.
    \[
        \left( \forall \varepsilon > 0 \quad \exists N \quad \forall n \geq N \quad d(x_n, x) < \varepsilon \right) 
    .\] 
    \[
    x = \lim_{n \to \infty} x_n
    .\] 
\end{definition}

\begin{theorem}
    In metric space convergent sequence has only one boundary.
\end{theorem}
\begin{proof}
    Let $\lim_{n \to \infty} x_n = x, \; \lim_{n \to \infty} x_n = y $.
    \[
    0 \leq d(x, y) \leq d(x, x_n) + d(x_n, y) \to 0
    .\] 
    \[
    \Rightarrow d(x, y) = 0 \Rightarrow x = y
    .\] 
\end{proof}

$\left( X, d_x \right), \left( Y, d_y \right) $ - metric spaces. $f : X \to Y$.

\begin{definition}
    \begin{enumerate}
        \item $f$ continuous in point $x_0 \in X$ if 
            \[
            x_n \to x_0 \text{ in $X$ } \Rightarrow f(x_n) \to f\left( x_0 \right) \text{ in $Y$ }
            .\] 
        \item $f$ continuous on $X$ if $f$ continuous in every point $x_0 \in X$.
    \end{enumerate}
\end{definition}

\begin{remark}
    $f$ continuous in $x_0 \in X$ then and only then if
    \[\forall \varepsilon > 0 \quad \exists \delta > 0 : d_x(x, x_0) < \delta \Rightarrow d_x(f(x), f(x_0)) < \varepsilon \]
\end{remark}

\begin{definition}
    \begin{enumerate}
        \item $f: X \to Y$ is called homeomorphism if $f$ is bijective, continuous and $f^{-1}$ is continuous.
        \item $f: X \to Y$ isometric if 
            \[
            d_y (f(x), f(x')) = d_x(x, x')
            .\] 
            Isometrie is always continuous.
    \end{enumerate}
\end{definition}

$x \in X, \;\; r > 0$
\begin{definition}
    Open ball \[
    \mathbb{B}(x, y) = \{ y \in X : d(y, x) < r \} 
    .\] 
\end{definition}
\begin{definition}
    Closed ball \[
    \overline{\mathbb{B}} = \{ y \in X : d(y, x) \leq r \} 
    .\] 
\end{definition}

Convergence can be rewritten using the last two definitions:
\[
x_n \to x \iff \forall \varepsilon > 0 \quad \exists N \quad
\forall n \geq N \quad x_n \in \mathbb{B}(x, \varepsilon)
.\] 

\begin{figure}[ht]
    \centering
    \incfig{convergence-in-terms-of-open-and-closed-ball-definitions}
    \caption{Convergence in terms of open and closed ball definitions}
    \label{fig:convergence-in-terms-of-open-and-closed-ball-definitions}
\end{figure}

\begin{definition}
    $A \subset X$. Point $x$ is tangent to set $A$ if \[
    \forall \varepsilon > 0 \quad \mathbb{B}(x, \varepsilon) \cap A \not= \varnothing
    .\] 
\end{definition}

\begin{example}
    $X = \mathbb{R}$, $A = (a, b)$. $a$ and $b$ are tangent to $A$. \\
    All elements from set $A$ are tangent to $A$. \\
    If there's some $\exists c > b$ then we can pick some ball around $c$
    of radius $r$. In that ball there would be no elements from A.
\end{example}
\begin{figure}[ht]
    \centering
    \incfig{tangent-point-example}
    \caption{Tangent point example}
    \label{fig:tangent-point-example}
\end{figure}

\begin{definition}
    $\overline{A} = \{ x \in X : x \text{ tangent to } A \} $ closure of set $A$.
\end{definition}

\begin{theorem}[Properties of closure]
    Set $A$ with closure $\overline{A}$ has the following properties:
    \begin{enumerate}
        \item $A \subset \overline{A}$
        \item $\overline{\overline{A}} = \overline{A}$ --- idempotence
        \item $A \subset B \Rightarrow \overline{A} \subset \overline{B}$
        \item $\overline{A \cup B} = \overline{A} \cup \overline{B}$
    \end{enumerate}
\end{theorem}
\begin{proof}
    \begin{itemize}
        \item[1] \begin{align*}
        x \in A \Rightarrow \mathbb{B}(x, \varepsilon) \cap A \not= \varnothing \text{ since it contains $x$ }
        .\end{align*}
    \item[3] \begin{align*}
        x \in \overline{A} \Rightarrow \mathbb{B}(x, \varepsilon) \cap A \not= \varnothing \Rightarrow \\
        \Rightarrow \mathbb{B}(x, \varepsilon) \cap B \not= \varnothing \Rightarrow 
        x \in \overline{B}
        .\end{align*}
        \item[2] \begin{align*}
            \overline{A} \subset \overline{\overline{A}} \\
            \text{ need to show that } \overline{\overline{A}} \subset \overline{A} \\
            x \in \overline{\overline{A}}, \varepsilon > 0. \quad
            \mathbb{B}(x, \varepsilon) \cap \overline{A} \not= \varnothing \\
            \text{ exists point } y \in \mathbb{B}(x, \varepsilon) \cap \overline{A}.
        .\end{align*}
        \begin{figure}[ht]
            \centering
            \incfig{eps-ball-y-in-eps-ball-x-with-point-from-a}
            \caption{Eps-Ball Y in eps-Ball x with point from A}
            \label{fig:eps-ball-y-in-eps-ball-x-with-point-from-a}
        \end{figure}

        Lets show that $\mathbb{B}(y, \varepsilon - d(x, y)) \subset \mathbb{B}(x, \varepsilon)$.
        \begin{gather*}
            z \in \mathbb{B}(y, \varepsilon - d(x, y)) \\
            \text{ for $z$ the following is met } d(z, y) < \varepsilon - d(x, y) \\
            \varepsilon > d(z, y) + d(y, x) \geq d(z, x) \Rightarrow
            z \in \mathbb{B}(x, \varepsilon) \\ 
            \mathbb{B}(y, \varepsilon - d(x, y)) \cap A \not= \varnothing \Rightarrow 
            \mathbb{B}(x, \varepsilon) \cap A \not= \varnothing \\
            \Rightarrow x \in \overline{A}.
        \end{gather*}

    \item[4] \[
    A \subset A \cup B \Rightarrow \overline{A} \subset \overline{A \cup B}; \; \overline{B} \subset \overline{A \cup B}
    \] 
    \[
    \Rightarrow \overline{A} \cup \overline{B} \subset \overline{A \cup B}
    .\] 
\begin{figure}[ht]
    \centering
    \incfig{for-stupid-idiots}
    \caption{For stupid idiots}
    \label{fig:for-stupid-idiots}
\end{figure}

    Need to prove $\overline{A \cup B} \subset \overline{A} \cup \overline{B}$

    (by contradiction) \\
    Let $x \in \overline{A \cup B}$ and $x \not\in \overline{A}$, $x \not\in \overline{B}$.
    \[
    \exists \varepsilon_1 > 0 : \mathbb{B}\left( x, \varepsilon_1 \right) \cap A = \varnothing
    .\] 
    \[
    \exists \varepsilon_2 > 0 : \mathbb{B}\left( x, \varepsilon_2 \right) \cap B = \varnothing
    .\] 
    \[
    \varepsilon = \min(\varepsilon_1, \varepsilon_2). \quad
    \mathbb{B}(x, \varepsilon) \cap \left( A \cup B \right) = \varnothing
    .\] 
    \[
    \Rightarrow \overline{A \cup B} = \overline{A} \cup \overline{B}
    .\] 
    \end{itemize}
\end{proof}


\begin{theorem}
    $x \in \overline{A} \iff$ in set A exists series $\left( x_n : n \geq 1 \right) $ that converges to $x$.
\end{theorem}
\begin{proof}
    \begin{itemize}
        \item[($\Rightarrow$)] Let $x\in \overline{A}$
            \[
            \forall \varepsilon > 0 \quad \mathbb{B}(x, \varepsilon) \cap A \not= \varnothing
            .\] 
            let $\varepsilon_n = \frac{1}{n} \quad$ \\
            $\forall n \geq 1$ exists point $x_n \in A \cap \mathbb{B}(x, \frac{1}{n}$ \\
            \[
            0 \leq d(x, x_n) < \frac{1}{n} \to 0. \quad \lim_{n \to \infty} x_n = x
            .\] 
        \item[($\Leftarrow$)] Let $\lim_{n \to \infty} x_n = x$ and $x_n \in A$. \\
            \[
            \forall \varepsilon > 0 \quad \exists N \quad \forall n \geq N \quad d(x_n ,x) < \varepsilon
            .\] 
            \[
            x_n \in \mathbb{B}(x, \varepsilon) \cap A \not= \varnothing
            .\] 
            \[
            \Rightarrow x \in \overline{A}
            .\] 
    \end{itemize}
\end{proof}

\begin{definition}
    A is dense in set B if $B \subset \overline{A}$ (any B element can be approached to elements of A)
\end{definition}
\begin{definition}
    A dense everywhere if $\overline{A} = X$.
\end{definition}
\begin{definition}
    Metric space $\left( X, d \right) $ is separable if exists dense everywhere countable set.
\end{definition}

\begin{example}
    \begin{enumerate}
        \item $\mathbb{R}$ - separable space. $\overline{\mathbb{Q}} = \mathbb{R}$
        \item $\mathbb{R}^{n} $ - separable space related to any metric $d_{p}, \; 0 < p \leq \infty$. $\overline{\mathbb{Q}^{n}} = \mathbb{R}^{n}$
        \item $X, d$ - discrete. $\mathbb{B}(x, \varepsilon) = \{ y : d(x, y) < \varepsilon \} $. But if $0 < \varepsilon < 1$ then
            \[
            \mathbb{B}(x, \varepsilon) \cap A \not= \varnothing \iff x \in A.
            .\] 
            \[
             \Rightarrow \overline{A} = A
            .\] 
        The only dense everywhere set is X.
    \item $C[a, b]$; $d(f, g) = \underset{t \in [a, b]}{\sup} \left| f(t) - g(t) \right| $ \\
        By Weierstrasse theorem $\forall f \in C[a, b] \quad \forall \varepsilon > 0$ exists polynomial
    \[ P(t) = a_0 + a_1 t + \ldots + a_d t^{d} : \underset{t \in [a, b]}{\sup}
    \left| f(t) - P(t) \right| < \varepsilon \]
    Dense everywhere set is set of polymonials with rational coefficients.

    \item $C_{b}(\mathbb{R}), d(f, g) = \underset{t \in \mathbb{R}}{\sup} \left| f(t) - g(t) \right| $ - is not separable metric space.

\begin{figure}[ht]
    \centering
    \incfig{line-for-example-of-not-separable-metric-space}
    \caption{Line for example of not separable metric space}
    \label{fig:line-for-example-of-not-separable-metric-space}
\end{figure}

\[ A \subset \mathbb{Z} \quad f_{A}(n) =
 \begin{cases}
     1 & n \in A \\
     0 & n \in \mathbb{Z} \backslash A
 \end{cases}
.\] 
\[
A \not= A' \quad n \in A \backslash A' \text{ or } n \in A' \backslash A
.\] 
\[
d(f_{A}, f_{A'}) = 1
.\] 
\[
\mathbb{B}(f_{A}, \frac{1}{2}) \cap \mathbb{B}(f_{A'}, \frac{1}{2}) = \varnothing
.\] 
In space $C_b(\mathbb{R})$ exists a continuum family of open balls that do not intersect.

If dense everywhere set exists than in every open ball must be an element of the one.
\begin{figure}[ht]
    \centering
    \incfig{family-of-open-balls-each-containing-an-element}
    \caption{Family of open balls each containing an element}
    \label{fig:family-of-open-balls-each-containing-an-element}
\end{figure}
    \end{enumerate}
\end{example}
