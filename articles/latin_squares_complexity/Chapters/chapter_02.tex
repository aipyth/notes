\chapter{Складність}
Доповнення часткових латинських квадратів до латинських квадратів є \emph{NP-complete}. У данному розділі наводиться доказ цього твердження у два кроки (за Колборном \cite{colbourn82}).

\section{Розбиття тричасткових графів на трикутники}

Приведемо зв'язок між доповненням та проблемою з теорії графів, розбиттям тридольних графів на трикутники та покажемо, що ця проблема є \emph{NP-complete}.

\begin{definition}
    Граф $G = (V, E)$ є тричастковим графом, якщо його множина вершин $V$ може бути розбита на три множини $V_1, V_2, V_3$, які індукують незалежну множину на цьому графі. Також можна сказати, що такий граф може бути розфарбований трьома кольорами так, що не існує двох сусідніх вершин  (поєднаних ребром) з однаковим кольором.
\end{definition}

\begin{definition}
    Розбиття на трикутники графа $G = (V, E)$ це розбиття множини $E$ на множини, що будуть мати три ребра, які формують трикутник (або утворюють граф $K_3$) в графі $G$. 
\end{definition}

\begin{definition}
    Граф $G = (V, E)$ є одноманітним якщо він є тричастковим графом з розбиттям $V = V_1 \sqcup V_2 \sqcup V_3$ та умовою на кожну вершину: кожна вершина з $V_i, i = \overline{1, 3}$ має однакову кількість сусідніх з двох інших множин $V_j, j = \overline{1,3}, j \not= i$ розбиття множини $V$.
\end{definition}

Тричастковий граф з розбиттям на трикутники повинен бути одноманітним.

Зв'язок між доповненням часткового латинського квадрата та розбиттям на трикутники тричасткового графа є досить близьким. Ми можемо побудувати граф $G(P)$ використовуючи частковий латинський квадрат $P$. Граф $G(P)$ будемо позначати як $G_P$ та такий граф матиме наступну побудову:
\begin{enumerate}
    \item $G_P = (V, E)$
    \item $V = V_r \cup V_c \cup V_e$ \\
        $V_r = \set{r_i \mid i \text{ рядок містить пустий квадрат }}$ \\
        $V_c = \set{c_j \mid j\text{ колонка містить пустий квадрат }}$ \\
        $V_e = \set{e_k \mid k\text{ елемент з'являєтсья у менш ніж $n$ квадратах }}$
    \item Маємо наступні правила включення ребра до множини $E$: \begin{enumerate}
        \item якщо $(i, j)$ клітинка є пустою, то включаємо ребро $(r_i, c_j)$
        \item якщо рядок $i$ не містить елемента $k$, то включаємо ребро $(r_i, e_k)$
        \item якщо стовпець $j$ не містить елемента $k$, то включаємо ребро $(c_j, e_k)$
        \end{enumerate}
\end{enumerate}

Граф $G_P$ має розбиття на трикутники тоді та тільки тоді якщо $P$ може бути доповнений.

\begin{theorem}
    \label{theorem:tripartite_triangulation_np_complete}
    Задача вирішення чи має даний тричастковий граф розбиття на трикутники є NP-complete.
\end{theorem}
Дану теорему можна доказати привівши певні модифікації у доведення Холіера, що задача трикутного завершення графів є NP повною.
Ці модифікації наведені у роботі Колборна \cite{colbourn82}.

\section{Доповнення часткового латинського квадрата NP-complete}
\emph{Використовуємо позначення минулого розділу.}

\begin{definition}
    Латинський каркас $LF(G; r, s, t)$ деякого тричасткового графу $G$ це матриця розміру $r \times s$. Кожна клітинка є або пустою, або елементом із множини $\set{1, \dots, t}$. Кожний рядок (колонка) містить елемент щонайбільше один раз. Також є певні правила для заповнення даної матриці:
    \begin{enumerate}
        \item Якщо $G$ має ребро $(r_i, c_j)$, то клітинка $(i, j)$ є пустою, інакше містить елемент з множини $\set{1. \dots, t}$.        \item Якщо $G$ має ребро $(r_i, e_k)$, то рядок $i$ не містить елемента $k$.
        \item Якщо $G$ має ребро $(c_j, e_k)$, то колонка $j$ не містить елемента $k$.
    \end{enumerate}
\end{definition}

Потрібно зауважити, що якщо $r = s = t$, то наш каркас є частковим латинським квадратом, який може бути доповнений тоді та тільки тоді якщо $G$ має розбиття на трикутники.

\begin{lemma}
    \label{lemma:exists_lat_frame_for_g}
    Для одноматітного тричасткового графа $G$ з $n$ вершинами існує латинський каркас $LF(G; n, n, 2n)$.
\end{lemma}

\begin{proof}
    Маємо матрицю $L$ розміру $n \times n$. Якщо $(r_i, c_j)$ є ребром $G$, то потрібно залишити $(i, j)$ клітинку пустою. Інакше заповнити значенням $1 + n + ((i + j) mod n)$. Маємо $L$ як $LF(G; n, n, 2n))$.
\end{proof}

Наведемо наступні леми без доведення (можна знайти у \cite{colbourn82}):
\begin{lemma}
    \label{lemma:add_column}
    Нехай $L$ це $LF(G; r, s, t)$ одноманітного тричастковго графа $G$.
    Позначимо $R(k)$ як разів появи елемента $k$ у $L$ плюс половина степені вершини $e_k$ графа $G$. Далі будь-коли $R(k) \geq r + s - t$ для будь-яких $1 \leq k \leq t$, $L$ може бути розширений до $LF(G; r, s+1, t)$ $L'$ у якому $R'(k) \geq r + (s+1) - t$ для будь-яких $1 \leq k \leq t$.
\end{lemma}

Ми використаймо цю лему \ref{lemma:add_column} для додавання колонок. Щоб додати рядки можна транспонувати матрицю та скористатися цією лемою.
Насправді Маршалл Голл показав, що латинський прямокутник $r \times s$ завжди може бути доповнений до латинського квадрата.

\begin{lemma}
    \label{lemma:latin_frame_extended}
    Латинський каркас $LF(G; r, s, s)$ для однорідного тричасткового графа $G$ може бути доповнений до латинького каркаса $LF(G; s, s, s)$.
\end{lemma}

Ці три наведені вище леми ведуть до:
\begin{theorem}
    \label{theorem:uniform_to_latin_frame_polymonial}
    При однорідному тричастковому графі $G$ з $n$ вершинами латинський каркас $LF(G; 2n,2n,2n)$ може бути отриманий за поліноміальний час.
\end{theorem}

\begin{proof}
    Спочатку ми створюємо $LF(G; n, n, 2n)$ за допомогою леми \ref{lemma:exists_lat_frame_for_g}.
    Використаймо послідовно лему \ref{lemma:latin_frame_extended} для утворення $LF(G; n, 2n, 2n)$.
    Далі транспонуємо цю матрицю та використаємо послідовно лему це раз отримуючи $LF(G; 2n, 2n, 2n)$.
    Даний процес займає поліноміальний час. 
\end{proof}

Отже ця теорема дає змогу привести задачу триангуляції (розбиття на трикутники) до доповнення латинських квадратів.

І, нарешті!!!
\begin{theorem}
    Задача рішення чи може латинський квадрат бути доповнений є $\mathcal{NP}$-complete.
\end{theorem}
\begin{proof}
    По перше, членство у класі NP є безпосереднім (it is immediate). По друге, потрібно показати повноту.
    Для цього зведемо триангуляцію тричасткових графів та теорему \ref{theorem:tripartite_triangulation_np_complete}, за якою ця задача є NP-complete.
    Маючи тричастковий граф із $n$ вершинами ми повинні встановити його однорідність.
    Якщо він не є однорідним, то і не є існує розбиття на трикутники. Якщо ж є однорідним, то використаймо теорему \ref{theorem:uniform_to_latin_frame_polymonial} для створення латинського каркаса $LF(G; 2n, 2n, 2n)$ за поліноміальний час. Цей латинський каркас і є також частковим латинським квадратом.
    Далі потрібно показати, що цей частковий латинський квадрат може бути доповнений лише у тому випадку, якщо $G$ має розбиття на трикутники.
    А це вже слідує з того, що за побудовою граф $G$ сконструйований з часткового латинського квадрату.
\end{proof}
