% arara: pdflatex
\documentclass[12pt,letterpaper]{article}
\usepackage{fullpage}
\usepackage[top=2cm, bottom=4.5cm, left=2.5cm, right=2.5cm]{geometry}
\usepackage{amsmath,amsthm,amsfonts,amssymb,amscd}
\usepackage{hyperref}
\usepackage{xcolor}
\usepackage{fancyhdr}
\usepackage{mathrsfs}

\usepackage{mathtools}
\DeclarePairedDelimiter{\ceil}{\lceil}{\rceil}

\usepackage{fontspec}

\setromanfont{PTSerif}[
    Path=./fonts/,
    Extension = .ttf,
    UprightFont = *-Regular,
    BoldFont = *-Bold,
    ItalicFont = *-Italic,
    BoldItalicFont = *-BoldItalic,
]

\hypersetup{%
  colorlinks=true,
  linkcolor=blue,
  linkbordercolor={0 0 1}
}

%%%%%%%%%%%%%%%%%%%%%%%%%%%%%%%%%%%%%%%%%%%%%%%%%%%%%%%%%%%%%%%%%%%%%%%%%%%%%%
\newcommand\hwnumber{2}
\newcommand\student{Ivan Zhytkevych}

%%%%%%%%%%%%%%%%%%%%%%%%%%%%%%%%%%%%%%%%%%%%%%%%%%%%%%%%%%%%%%%%%%%%%%%%%%%%%%
\pagestyle{fancyplain}
\headheight 35pt
\lhead{\student}
\chead{\textbf{\Large CT Homework \hwnumber}}
\rhead{\today}
\lfoot{}
\cfoot{}
\rfoot{\small\thepage}
\headsep 1.5em

%%%%%%%%%%%%%%%%%%%%%%%%%%%%%%%%%%%%%%%%%%%%%%%%%%%%%%%%%%%%%%%%%%%%%%%%%%%%%%
\begin{document}
\tableofcontents
\newpage
%%%%%%%%%%%%%%%%%%%%%%%%%%%%%%%%%%%%%%%%%%%%%%%%%%%%%%%%%%%%%%%%%%%%%%%%%%%%%%%
\section{Problem 2.4}

Нехай стандартна багатострічкова МТ виконує задачу за час $t$.

Порівняємо виконання тієї ж задачі на стандартній багатострічковій МТ (МТ1) та на
довільній багатоканальній МТ (МТ2), на якій зчитувальні пристрої рухаються одночасно:
\begin{quotation}
    Якщо на МТ1 зчитувальний пристрій на якійсь стрічці $i$ рухається вправо,
    а на іншій $j : j \neq i$ рухається вліво, то для МТ2 ці дії на обох стрічках
    на будуть проводитися паралельно. Натомість потрібно буде закінчити дію на
    стрічці $i$, не змінюючи стрічку $j$, та повернутися до зміни стрічки $j$,
    не змінюючи стрічку $i$.
\end{quotation}

Тобто для певної дії $\zeta$, яка займає $k_{\zeta}$ тактів на МТ1 на МТ2 вона займе
$k_{\zeta} + r_{\zeta} + k_{\zeta} = 2k_{\zeta} + r_{\zeta}$, де $r_{\zeta}$
це кількість тактів на повернення.

Таким чином для МТ1, що виконує задачу за $t = \sum k_{\zeta}$ тактів на МТ2
ця задача займе:
\[ \sum_{\zeta} (2k_{\zeta} + r_{\zeta}) = 2t + R \]

%%%%%%%%%%%%%%%%%%%%%%%%%%%%%%%%%%%%%%%%%%%%%%%%%%%%%%%%%%%%%%%%%%%%%%%%%%%%%%%
\section{Problem 2.5a}

\begin{enumerate}
    \item Маємо дві стрічки: вхідну та вихідну.

    \item Проходимо вхідну стрічку записуючи результат проходу в вихідну:
        \begin{enumerate}
            \item Символ $c$ та $b$ у стані $q_0$ записуємо без змін.
            \item На символі $a$ переходимо у новий стан, що буде означати очікування
                символу $b$ для заміни $ab \rightarrow c$ (при переході на новий
                описаний стан не записується символ $a$ на вихідну стрічку)
            \item Якщо на стані очікування зчитуємо символ $b$, то записуємо символ $c$
                та рухаємося далі. Інакше записуємо символ $a$ та поточний зчитанний.
        \end{enumerate}
\end{enumerate}

%%%%%%%%%%%%%%%%%%%%%%%%%%%%%%%%%%%%%%%%%%%%%%%%%%%%%%%%%%%%%%%%%%%%%%%%%%%%%%%
\section{Problem 2.5b}

\begin{enumerate}
    \item Дві стрічки: вхідна (зчитуємо) та вихідна (записуємо).

    \item Копіюємо вхідне слово на вихідну стрічку.

    \item Повертаємося на вхідній стрічці на початок.

    \item Знову копіюємо вхідне слово на вихідну стрічку (зчитувальний пристрій
        вихідної стрічки при поверненні на кроці 3 залишається на місці, тобто на
        першому пустому символі після слова)

\end{enumerate}

%%%%%%%%%%%%%%%%%%%%%%%%%%%%%%%%%%%%%%%%%%%%%%%%%%%%%%%%%%%%%%%%%%%%%%%%%%%%%%%
\section{Problem 2.5c}

\begin{enumerate}
    \item Дві стрічки: вхідна (зчитування) та вихідна (запис).
    \item Записуємо нуль із вхідної на вихідну.
    \item Копіюємо всі одиниці із вхідної, поки не зустрінемо нуль та записуємо його.
    \item Повертаємося на вихідній стрічці на початок.
    \item Копіюємо решту вхідного слова, рухаючись вліво по вихідній стрічці.

\end{enumerate}

%%%%%%%%%%%%%%%%%%%%%%%%%%%%%%%%%%%%%%%%%%%%%%%%%%%%%%%%%%%%%%%%%%%%%%%%%%%%%%%
\section{Problem 2.5d}

\begin{enumerate}
    \item Маємо чотири стрічки: вхідну, стрічку для першого числа, стрічку для
        другого числа, вихідну стрічку.
    \item Копіюємо все із вхідної стрічки до символу $2$ на стрічку для першого числа.
    \item Копіюємо все з символу $2$ до пустого символу на стрічку для другого числа.
    \item Починаємо додавання двох чисел із стрічок для двох чисел на вихідну стрічку:
        \begin{enumerate}
            \item Переводимо зчитувальні пристрої стрічок двох чисел на кінець вправо
                цих чисел.
            \item Зчитувальні пристрої на стрічках чисел рухаються одночасно вліво.
            \item Додаємо два поточних символа за такими умовами:
                \begin{enumerate}
                    \item Якщо 1 0, 0 1, 1 \#, \# 1, то записуємо 1 та рухаємося далі.
                    \item Якщо 0 0, то записуємо 0 та рухаємося далі без переповнення.
                    \item Якщо 1 1, то переходимо в стан переповнення та записуємо 0.
                    \item Якщо 1 0, 0 1, 1 \#, \# 1 та стан переповнення, то
                        записуємо 1, переходимо в нормальний стан
                    \item Якщо 0 0 та стан переповнення, то записуємо 1 та переходимо
                        в нормальний стан.
                    \item Якщо 1 1 та стан переповнення, то записуємо 1 та залишаємося
                        в стані переповнення.
                    \item Якщо \# \# та стан переповнення, то записуємо 1 та завершуємо роботу.
                    \item Якщо \# \# та в не в стані переповнення, то завершуємо роботу.
                \end{enumerate}
        \end{enumerate}
\end{enumerate}

%%%%%%%%%%%%%%%%%%%%%%%%%%%%%%%%%%%%%%%%%%%%%%%%%%%%%%%%%%%%%%%%%%%%%%%%%%%%%%%

\section{Problem 2.6}

\begin{enumerate}
    \item \textbf{$n$}

        Побудована машина Тюрінга для доказу конструктивності за часом функції $n$
        копіює вхідне слово із одиниць $1^n$ з вхідної стрічки на вихідну за $n$ тактів.

    \item \textbf{$n log n$}

        Побудована машина Тюрінга для доказу конструктивності за часом функції $n log n$
        перетворює вхідне слово із вхідної стрічки у двійкове на додаткову стрічку №1.
        Це перетворення описується так: на кожний прочитаний вхідний символ машина
        перезаписує двійкове число на стрічці №1. Максимальна кількість операцій для
        перезапису на вхідному слові $1^n$ буде $log n$. Оскільки цей перезапис буде
        проводитися після прочитання кожного вхідного символу, то можемо грубо оцінити
        його в $O(n long n)$ операцій. Далі вхідне число з стрічки №1 записується на
        вихідну стрічку за $log n$ разів, де лічильником слугує слово на стрічці №1.
        Отримали таку машину Тюрінга, що перетворює $1^n \rightarrow 1^{n log n}$.


    \item \textbf{$n^2$}

        Машина Тюрінга для доказу конструктивності за часом функції $n^2$ описана таким
        чином:
        \begin{enumerate}
            \item Маємо дві стрічки: вхідна та вихідна.
            \item Копіюємо слово з вхідної стрічки на вихідну знову таким чином,
                що після $k$ копіювань на вихідній стрічці маємо слово $1^{kn}$.
            \item Повторюємо кроки 2 поки можемо.
        \end{enumerate}
        Крок 2 повторюється $n$ разів, копіювання займає $n$ тактів, а отже дана
        МТ працює за $t(n) = n^2$.

    \item \textbf{$2^n$}

        Будуємо машину Тюрінга із трьома стрічками. Записуємо символ '1' на додаткову
        та вихідну стрічки. Потім на кожний зчитувальний символ першої стрічки дописуємо
        все слово з додаткової стрічки на вихідну та попереднє слово з вихідної на
        додаткову. Кожне $k$ копіювання займає $O(2^k)$.

\end{enumerate}


%%%%%%%%%%%%%%%%%%%%%%%%%%%%%%%%%%%%%%%%%%%%%%%%%%%%%%%%%%%%%%%%%%%%%%%%%%%%%%%
\section{Problem 2.7}

Нехай існує така машина Тюрінга, яка працює $\ceil{log n}$ тактів та для вхідного 
слова $1^n$ повертає $1^{log n}$.

Розглянемо два слова довжини $n_1$ та довжини $n_2 = n_1 + 1$.
Слово може бути зчитано за $n$ тактів. Але ми маємо лише $\ceil{log n}$ тактів.
Таким чином для слів довжини $n_1$ та $n_2$ машина Тюрінга буде зчитувати лише
$\ceil{log n_1}$ та $\ceil{log n_2} = \ceil{log (n_1 + 1)} = \ceil{log n_1}$, а отже
видавати однаковий результат. Маємо протиріччя.

%%%%%%%%%%%%%%%%%%%%%%%%%%%%%%%%%%%%%%%%%%%%%%%%%%%%%%%%%%%%%%%%%%%%%%%%%%%%%%%%%%%
\section{Problem 2.9}

Нехай існують такі машини Тюрінга:
\begin{enumerate}
    \item \[ M_{f}: 1^{n} \rightarrow 1^{f(n)} \]
        Працює за $O(f(n))$. Має $l_f$ стрічок.

    \item \[ M_{g}: 1^n \rightarrow 1^{g(n)} \]
        Працює за $O(g(n))$. Має $l_g$ стрічок.
\end{enumerate}

\begin{enumerate}
    \item $h_1 = f(n) + g(n)$
        
        Будуємо машину Тюрінга $M_{h_1}$, що має $l_{h1} = l_f + l_g - 2$ стрічок.
        Спочатку працює підмашина $M_f$ на перших $l_f$ стрічках, що займе $O(f(n))$
        тактів та результат буде записано на $l_f$ стрічці. Потім працює підмашина
        $M_g$, у яка працює з $l_f + 1$ до $l_f + l_f - 2$ стрічки та на першій
        стрічці, та займає $O(g(n))$ часу. Вихідна стрічка це $l_f + l_g - 2$
        стрічка. $M_{h_1}$ працює $O(f(n) + g(n))$. Отже $f(n) + g(n)$  конструктивною.

    \item $h_2 = f(n) g(n)$

        Будуємо аналогічну пунту 1 МТ, окрім:

        На кінець машина записує на вихід слово $1^{f(n)}$, отримане $M_f$, $g(n)$
        разів, пробігаючи слово $1^{f(n)}$ $g(n)$ разів. Отримуємо час роботи $O(f(n)g(n))$

    \item $h_3 = g(f(n))$

        Будуємо машину Тюрінга аналогічну пункту 1, окрім того, що підмашині $M_g$ на
        вхід буде пдаватися слово не з першої стрічки, а зі стрічки $l_f$. Тобто ця
        підмашина буде працювати на $l_f$ - $l_f + l_g - 2$ стрічках.

    \item $h_4 = 2^{f(n)}$

        Є композицією конструктивних функцій, що є конструктивною функцією.

    \item $h_5 = f(n)^{g(n)}$

        Будуємо машину Тюрінга аналогічно, але з правилами як у задачі 2.6.4.
        На вихідну стрічку записується слово з додаткової $f(n) - 1$ разів.

        Таким чином $f(n)^n$ буде конструктивною за часом, а отже конструктивною буде і
        $f(n)^{g(n)}$.

        
\end{enumerate}

%%%%%%%%%%%%%%%%%%%%%%%%%%%%%%%%%%%%%%%%%%%%%%%%%%%%%%%%%%%%%%%%%%%%%%%%%%%%%%%%%%%

\section{Problem 2.11}

Візьмемо на приклад машину Тюрінга з задачі  2.6.2. Відкинемо третю стрічку, додаткова
буде вихідною. Залишається лише перетворення числа з унарного на бінарний запис.
Замінюємо всі нулі на одиниці та отримуємо $1^{log n}$. Пам'яті використано $\ceil{log n}$

%%%%%%%%%%%%%%%%%%%%%%%%%%%%%%%%%%%%%%%%%%%%%%%%%%%%%%%%%%%%%%%%%%%%%%%%%%%%%%%%%%%

\end{document}

