\documentclass[12pt,letterpaper]{article}
\usepackage{fullpage}
\usepackage[top=2cm, bottom=4.5cm, left=2.5cm, right=2.5cm]{geometry}
\usepackage{amsmath,amsthm,amsfonts,amssymb,amscd}
\usepackage{hyperref}
\usepackage{xcolor}
\usepackage{fancyhdr}
\usepackage{mathrsfs}

\usepackage{mathtools}
\DeclarePairedDelimiter{\ceil}{\lceil}{\rceil}

\usepackage{fontspec}

\setromanfont{PTSerif}[
    Path=./fonts/,
    Extension = .ttf,
    UprightFont = *-Regular,
    BoldFont = *-Bold,
    ItalicFont = *-Italic,
    BoldItalicFont = *-BoldItalic,
]

\hypersetup{%
  colorlinks=true,
  linkcolor=blue,
  linkbordercolor={0 0 1}
}

%%%%%%%%%%%%%%%%%%%%%%%%%%%%%%%%%%%%%%%%%%%%%%%%%%%%%%%%%%%%%%%%%%%%%%%%%%%%%%
\newcommand\hwnumber{5}
\newcommand\student{Ivan Zhytkevych}

%%%%%%%%%%%%%%%%%%%%%%%%%%%%%%%%%%%%%%%%%%%%%%%%%%%%%%%%%%%%%%%%%%%%%%%%%%%%%%
\pagestyle{fancyplain}
\headheight 35pt
\lhead{\student}
\chead{\textbf{\Large Complexity Homework \hwnumber}}
\lfoot{}
\cfoot{}
\rfoot{\small\thepage}
\headsep 1.5em

\NewDocumentCommand{\set}{o m}{%
  % <code>
  \IfNoValueTF{#1}
    {\{#2\}}
    {\{#1 \mid #2\}}%
  % <code>
}

%%%%%%%%%%%%%%%%%%%%%%%%%%%%%%%%%%%%%%%%%%%%%%%%%%%%%%%%%%%%%%%%%%%%%%%%%%%%%%
\begin{document}
% \tableofcontents
% \newpage
%%%%%%%%%%%%%%%%%%%%%%%%%%%%%%%%%%%%%%%%%%%%%%%%%%%%%%%%%%%%%%%%%%%%%%%%%%%%%%%
\section{Problem 5.1}

\begin{itemize}
    \item[$L_1 \cup L_2$] Будуємо МТ, яка буде запускати вирішувачі мов $L_1, L_2$ паралельно, та повертати reject якщо обидві вирішувачі не прийняли слово, інакше --- accept. Дана МТ вирішує об'єднання та працює за поліноміальний час.
    \item[$L_1 \cap L_2$] Аналогічно. Запускаємо МТ, якщо обидва вирішувачі повертають 1, то також повертаємо 1, інакше --- 0. Працюємо за поліноміальний час та вирішуємо перетин.
    \item[$L_1 L_2$] Для конкатенації наша МТ повинна розглядати всі префікси та постфікси слова, не забуваючи і про порожні. Маємо скінченну кількість таких префіксів та постфіксів, передаємо їх на вхід до вирішувачів $L_1$ та $L_2$. Результат повертаємо за аналогією до попереднього пункту. Працює за поліноміальний час та вирішує конкатенацію.
    \item[$L_1^R$] Наша МТ пише слово у зворотньому порядку, передає на вхід до вирішувача $L_1$ та повертає результат вирішувача. Працюємо за поліноміальний час.
    \item[$L_1^*$] Аналогічно до пункту з конкатенацію, окрім того, що наша МТ повинна послідовно розділяти слово на 0, 1, 2, ... частин та перевіряти кожну на вирішувачі $L_1$. Також потрібно зауважити, що будемо мати скінченну кількість таких розділів слова.
\end{itemize}

%%%%%%%%%%%%%%%%%%%%%%%%%%%%%%%%%%%%%%%%%%%%%%%%%%%%%%%%%%%%%%%%%%%%%%%%%%%%%%%
\section{Problem 5.2}
\begin{itemize}
    \item[\textbf{a}] Складність алгоритму факторизації --- $O(n^2)$. Зчитування слова --- знехтовно мале. Отже $O(n^2)$, $L_{UFACTOR} \in P$.
    \item[\textbf{b}] Алгоритм Дейкстри --- $O(n^2) \Rightarrow L_{PATH} \in P$.
    \item[\textbf{c}] Зводимо задачу до пошуку найкоротшого шляху (алг. Дейкстри). При запуску із кожної вершини графа маємо складність $O(n^3) \Rightarrow L_{CONNECTED} \in P$.
\end{itemize}

%%%%%%%%%%%%%%%%%%%%%%%%%%%%%%%%%%%%%%%%%%%%%%%%%%%%%%%%%%%%%%%%%%%%%%%%%%%%%%%
\section{Problem 5.5}

$L_1 = HALT$. $L_2 = L_1 \cup \set{0,1}$. $L_2^* \in P$. $L_2 \not\in P$.

%%%%%%%%%%%%%%%%%%%%%%%%%%%%%%%%%%%%%%%%%%%%%%%%%%%%%%%%%%%%%%%%%%%%%%%%%%%%%%%%%%%
\section{Problem 5.8}

Нехай є така мова $L \in NP$. Тоді $\forall L_1 \in NP : L_1 \leq_D L$. $\Rightarrow L_1 \in coNP \Rightarrow NP \subseteq coNP$.

$\forall L_2 \in coNP : L_2 \leq_p L \Rightarrow L_2 \in NP \Rightarrow coNP \subseteq NP$.

$\Rightarrow coNP = NP$.

%%%%%%%%%%%%%%%%%%%%%%%%%%%%%%%%%%%%%%%%%%%%%%%%%%%%%%%%%%%%%%%%%%%%%%%%%%%%%%%%%%%
\section{Problem 5.10}

Побудуємо зведення $L_2 \leq_p L_1$.
Нехай існує така недетермінована МТ $M$, що розв'язує мову $L_2$. Якщо $M(x) = 1$, то
$x$ без префікса $1$ належить мові $L_1$. Якщо $M(x) = 0$, то $x$ без префікса $1$ не належить $L_1$.

%%%%%%%%%%%%%%%%%%%%%%%%%%%%%%%%%%%%%%%%%%%%%%%%%%%%%%%%%%%%%%%%%%%%%%%%%%%%%%%%%%%
\section{Problem 5.12}

\begin{itemize}
    \item[\textbf{a}] Припустимо, що $\exists L_{BigCycle} \leq_p L_{HAMPATH} \in NPC$.
        Домножимо граф на $n$ вершин без ребер, а потім припустимо, що МТ, яка розпізнає
        $L_{BigCycle}$ поверне $1$. Тоді визідний граф буде мати цикл із $n$ вершин. $\Rightarrow \in NPC$.
    \item[\textbf{b}] $L_{SmallCycle} = co L_{BigCycle} \Rightarrow L_{SmallCycle} \in NP
        \Rightarrow L_{SmallCycle} \in coNP.$
    \item[\textbf{c}] Зведемо $SAT \rightarrow L_{TAUTOLOGY}$.

        $L_{TAUTOLOGY}(\overline{\varphi}) = 1\Rightarrow SAT(\varphi) = 0$ \\
        $L_{TAUTOLOGY}(\overline{\varphi}) = 0 \Rightarrow SAT(\varphi) = 1$\\
        $L_{TAUTOLOGY} \in coNP$.

    \item[\textbf{d}] Зводимо до SAT додаючи ще одну модель:
        \[ \varphi(x_1, x_2, \dots, x_n) \rightarrow \overline{\varphi}(x_1, x_2, \dots, x_n, x_{n+1}) \land (x_1 \land x_2 \land \dots \land x_{n+1}) \]
    \item[\textbf{e}] Будуємо зведення до $L_{HAMPATH}$. Покладемо $k = n - 1$. Тоді шлях існуватиме, якщо $L_{restrPATH}$ поверне 1, інакше ні.
    \item[\textbf{f}] $\forall L_1 \in NP : L_1 \leq_p HALT$.\\
        Будуємо ДМТ $M$, яка буде працювати за поліноміальний час. Модифікуємо її так, що вона буде приймати лише якесь вхідне слово. $\Rightarrow HALT \in NP\text{-hard}$.
\end{itemize}

%%%%%%%%%%%%%%%%%%%%%%%%%%%%%%%%%%%%%%%%%%%%%%%%%%%%%%%%%%%%%%%%%%%%%%%%%%%%%%%%%%%
\end{document}

