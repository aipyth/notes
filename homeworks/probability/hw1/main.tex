% !TEX root = main.tex
\documentclass{homework}

\usepackage{enumitem}

\title{Probability Theory Homework №1}
\author{Ivan Zhytkevych}

\begin{document}

\maketitle

%%%%%%%%%%%%%%%%%%%%%%%%%%%%%%%%%%%%%%%%%%%%%%%%%%%%%%%%%%%%%%%%%%%%%%%%%%%%%%%%%%%%%%%%%%%%%%
\exercise[1.13]
Let's define set \( A_i \) that contains all permutations of elements in set \( S = \set{ 1, 2, \dots{}, n } \) where \( i \)-th element is on its place.
\[ A_i = \set{ a_1 \dots{} a_i \dots{} a_n : a_i \in S ; \; a_i \; \text{is on its place} } \]

Let's find the cardinality of this set.
From the definition of the set we know that every element \( \)  has \( a_i \) that is in place
\textit{i}. But the positions of other \( a_j ,\; j \neq i \) doesn't matter, so take number of
permutations of these `other \(a_j\)` which is \( (n-1)! \).

\[ \abs{A_i} = (n-1)! \]

Using this built object we can find \( \omega_n \) -- number of set \( S \)
permutations where no number stays in it's place.

\[ \omega_n = P_t - P_s \]

\( P_t \) is total number of permutations of \textit{S} and \( P_s \) is number of permutations of
\textit{S} where exists an element that is in it's place.

Obviously:
\[ P_t = n! \]

And what about \( P_s \)? Let's construct a union of all \(A_i\)

\( \bigcup_{i=1}^{n} A_i \) such a union contains all permutations where an element that is in it's
place exists.

\[ \abs{\bigcup_{i=1}^{n} A_i } = \sum_{i=1}^{n}{\abs{A_i}} - \sum_{i<j}{\abs{A_i \cap A_j}} + \dots
+ (-1)^{n-1}\abs{\bigcap_{i=1}^{n}{A_i}}\]

Here we got a problem. We don't know the cardinality of intersections.

The intersection of two sets \(A_i\) and \(A_j\) contains all permutations where \textit{i}-th and
\textit{j}-th elements are in it's place. The way cardinality of \(A_i\) found we know that two
elemets stay in their positions and others can go anywhere. Find number of permutations with two
elements stay in place:
\[ \abs{A_i \cap A_j} = (n-2)! \]

Similarly for any number of intersections:
\[ \abs{\bigcap_{i=1}^{k}{A_i}} = (n-k)! \]

Last step for \textbf{Inclusion–exclusion principle} -- we need to do something with these sums.

When we used this principle the exact elements and distinction in sum's arguments were not
mentioned. So we only calculate numbers as long as cardinality is found using the statements
above.

So, the number of ways to pick one element from a set \(S\) is \( C_{n}^{1} \), for two elements --
\( C_{n}^{2} \) and so on:

\[ \abs{\bigcup_{i=1}^{n} A_i } = C_{n}^{1} (n-1)! - C_{n}^{2} (n-2)! + \dots + (-1)^{n-1}
    C_{n}^{n} = \]
\[ = \frac{n!}{1! (n-1)!} (n-1)! - \frac{n!}{2! (n-2)!} (n-2)! + \dots + (-1)^{n-1} = \]
\[ = n! - \frac{n!}{2!} + \dots + (-1)^{n-1} \]

Finally find \(\omega_n\):
\[ \omega_n = P_t - P_s = n! - n! + \frac{n!}{2!} - \frac{n!}{3!} + \dots + (-1)^{n} = \]
need to mention that here used \( - (-1)^{n-1} = (-1)^{n} \)
\[ = \frac{n!}{2!} - \frac{n!}{3!} + \dots + (-1)^{n} \]

Last step -- find \( p_n = \ddfrac{\omega_n}{n!} \):

\[ p_n = \ddfrac{\omega_n}{n!} = \frac{n!}{2! n!} - \frac{n!}{3! n!} + \dots + \frac{(-1)^{n}}{n!} =
\frac{1}{2!} - \frac{1}{3!} + \dots + \frac{(-1)^{n}}{n!} \]


%%%%%%%%%%%%%%%%%%%%%%%%%%%%%%%%%%%%%%%%%%%%%%%%%%%%%%%%%%%%%%%%%%%%%%%%%%%%%%%%%%%%%%%%%%%%%%%%
\exercise[1.14]

First of all, define set of all learners, learners attending math class and 
learners attending physics class:

\[ A = \set{ a : \text{ \textit{a} -- learner } } ; \; \abs{A} = 35 \]
\[ M = \set{ m : \text{ \textit{b} -- learner and attends math class }} ; \; \abs{M} = 20 \]
\[ P = \set{ p : \text{ \textit{p} -- learner and attends physics class }}  ; \; \abs{P} = 11 \]
\[ N = \set{ n : \text{ \textit{n} -- learner and attends no classes } }; \; \abs{N} = 10 \]
\[ A = M \cup P \cup N \]

Find cardinality of \(A\):
\[ \abs{A} = \abs{M \cup P \cup N} \]

Use \textbf{Inclusion–exclusion principle}
\[ \abs{A} = \abs{M} + \abs{P} + \abs{N} - \abs{M \cap P} - \abs{M \cap N} 
- \abs{P \cap N} + \abs{M \cap P \cap N} \]

The \(N\) set is defined as set of non-attenders so there cannot be any intersections with sets
\(M\) and \(P\).

\[ \abs{A} = \abs{M} + \abs{P} + \abs{N} - \abs{M \cap P} \]
\[ 35 = 20 + 11 + 10 - \abs{M \cap P} \]
\[ \abs{M \cap P} = 20 + 11 + 10 - 35 = 41 - 35 = 6 \]

Number of students who attend math group and physics group -- 6.

And number of students attending only math group:
\[ \abs{M} - \abs{M \cap P} = 20 - 6 = 14 \]

%%%%%%%%%%%%%%%%%%%%%%%%%%%%%%%%%%%%%%%%%%%%%%%%%%%%%%%%%%%%%%%%%%%%%%%%%%%%%%%%%%%%%%%%%%%%%%%%
\exercise*[1.15]

\( \abs{X} = n; \; \abs{Y} = m \)

Define such relation:
\[ \mathit{f}: X \to Y \; \text{or} \; \mathit{f} \subseteq X \times Y \]
\(\mathit{f}\) is fully defined and functional:
\[  \forall x \in X \; \exists! y \in Y: \mathit{f}(x) = y \]

And let's also define set of permutations where each element would represent \(\mathit{f}\):
\[ M = \set{ 
        \begin{pmatrix}
            x_1    & x_2    & \dots & x_n \\
            M(x_1) & M(x_2) & \dots & M(x_n)
        \end{pmatrix}
        : x_i \in X, \; M(x_i) \in Y
} \]

\begin{enumerate}
    \item \textbf{Number of arbitrary functions}

        No additional conditions for set \textit M. Find cardinality of \textit M:
        for each element \textit x from \textit X we can map some element \textit y from \textit Y
        so there are \textit m ways to choose \textit y
        \[ \abs M = m ^ n \]


    \item \textbf{Number of injective functions}
        
        Set injection condition on set \textit M:
        \[ \forall x_1, x_2 \in X \;\;\; x_1 \neq x_2 \Rightarrow M(x_1) \neq M(x_2) \]

        That condition means distinct values \( M(x_i) \) for each pair of \( x \). Map the first
        \(x_1\) to some value from \(Y\) out of \(m\) possible variants and for the next \( x_2 \) number of
        such values to choose would be \(m-1\), for \(x_i\) the number of remaining elements in
        \( Y \backslash \set{\textit{chosen y's}} \) would be \(m-i+1\). The number of such permutations:
        \[ \abs M = m \cdot (m-1) \cdot (m-2) \dots (m-n+1) = \frac{m!}{(m-n)!}\]


    \item \textbf{Number of bijective functions}

        Add bijection condition on set \textit M from the previous subtask:
        \[ \forall y \in Y \; \exists x \in X : M(x) = y \]

        If \(n = m\) the number of left elements \( y : \nexists x \in X \; M(x) = y \) is \( 0 \).
        So number of such permutations:
        \[ \abs M = m! = n! \]


\end{enumerate}



%%%%%%%%%%%%%%%%%%%%%%%%%%%%%%%%%%%%%%%%%%%%%%%%%%%%%%%%%%%%%%%%%%%%%%%%%%%%%%%%%%%%%%%%%%%%%%%%
\exercise*[1.16]

\begin{enumerate}
    \item \textbf{Digits 0, 1, 2, 3, 4, 5}

        Define set of digits
        \[ D = \set{0, 1, 2, 3, 4, 5}; \; \abs{D} = 6 \]
        Define set of 3-digit numbers of digits of set D (the first digit cannot be 0):
        \[ A = \set{ a_1 a_2 a_3 : a_2, a_3 \in D; \; a_1 \in D \backslash \set{0}  } \]
        Use product rule to compute cardinality of set \(A\) which is number of such numbers
        \[ \abs{A} = 5 \cdot 6 \cdot 6 \]

    \item \textbf{Digits 0, 1, 2, 3, 4, 5 can be used only once}

        Define set of digits
        \[ D = \set{0, 1, 2, 3, 4, 5}; \; \abs{D} = 6 \]
        Define set of 3-digit numbers of digits of set D (the first digit cannot be 0):
        \[ A = \set{ a_1 a_2 a_3 : a_2, a_3 \in D, \; a_1 \in D \backslash \set{0}, \; a_i \neq
        a_j \;\; \forall i, j: i \neq j } \]
        Number of required numbers:
        \[ \abs{A} = 5 \cdot 5 \cdot 4 \]

\end{enumerate}


%%%%%%%%%%%%%%%%%%%%%%%%%%%%%%%%%%%%%%%%%%%%%%%%%%%%%%%%%%%%%%%%%%%%%%%%%%%%%%%%%%%%%%%%%%%%%%%%
\exercise*[1.17]

Using set of digits:
\[ D = \set{ 0, 1, 2, 3, 4, 5, 6, 7, 8, 9 } \]

Set of five-digit numbers divisible by 5 (the first digit cannot be 0):
\[ S = \set{ a_1 a_2 a_3 a_4 a_5 : a_1 \in D \backslash \set{0}, \; a_i \in D \;\; i =
\overline{2, 4}, \; a_5 \in \set{0, 5} } \]

We define \( a_5 \in \set{0, 5} \) in order to make a number divisible by 5.
So number of required numbers:
\[ \abs{S} = 9 \cdot 10^{3} \cdot 2 \]


%%%%%%%%%%%%%%%%%%%%%%%%%%%%%%%%%%%%%%%%%%%%%%%%%%%%%%%%%%%%%%%%%%%%%%%%%%%%%%%%%%%%%%%%%%%%%%%%
\exercise*[1.18]

Set of buttons:
\[ B = \set{ 1, 2, 3, 4, 5 } \]

\( w \) is set of two buttons pressed simultaneously (the order doesn't matter). Let's define a set of such sets
and find it's cardinality:
\[ W = \set{ w = \set{ w_1, w_2 } : w_1, w_2 \in B } \]
\[ \abs W = C_{\abs{B}}^{\abs{w}} = C_{5}^{2} = \frac{5!}{2! (5-2)!} = \frac{3 \cdot 4 \cdot 5}{2 \cdot
3} = 10 \]

Set of combinations:
\[ A = \set{ (w, a_1, a_2, a_3) : w \in W, \; a_1, a_2, a_3 \in B \backslash w } \]

Let's find cardinality of \(A\):

Elements \(a_1, a_2, a_3\) can take values only from a set of 3 items because 
\( \abs B - \abs w = 3\).
So number of such combinations of \( a_1, a_2, a_3 \) is \( 3! = 6 \).

Using product rule we find the cardinality of \( A \) -- number of code combinations:
\[ \abs A = \abs W \cdot 6 = 10 \cdot 6 = 60 \]

If last 3 buttons to press can be repeated:
\[ 10 \cdot 3^3 \]

If first 2 buttons do not stay pressed and can be pressed later:
\[ 10 \cdot 5^3 \]


%%%%%%%%%%%%%%%%%%%%%%%%%%%%%%%%%%%%%%%%%%%%%%%%%%%%%%%%%%%%%%%%%%%%%%%%%%%%%%%%%%%%%%%%%%%%%%%%
\exercise*[1.19]

Set of cards:
\[ C = \set{ c : c \text{ -- card} } \]

Suits:
\[ C_1 = \set{ c : c \text{ -- card of suit 1 } } \]
\[ C_1 = \set{ c : c \text{ -- card of suit 1 } } \]
\[ C_1 = \set{ c : c \text{ -- card of suit 1 } } \]
\[ C_1 = \set{ c : c \text{ -- card of suit 1 } } \]

\begin{enumerate}[label=\alph*]
    \item Set of such combinations:
        \[ A = \set{ a_1 a_2 a_3 a_4 a_5 a_6 : a_1 \text{ -- king of spades}, a_i \in C
                \; i=\overline{2, 6}  } \]
                \[ \abs A = 1 \cdot C_{51}^{5} \]

    \item Number of picked card 6 but number of suits is 4 so in such combinations the repeat of
        suit representers is inevitable. There must be 1 card from each suit and other two can be of
        any suits. They may be of one suit or two different. Let's visualize these two variants of
        combinations:
        \[ 1: a b c d a a \]
        \[ 2: a b c d a b \]

        Calculate final result as sum of these variants configurations:
        
        For the first one we pick three card of one suit and then 3 cards of other 3 suits.
        \[ C_4^1 \cdot C_{13}^3 \cdot (C_{13}^1)^3 \]

        For the second one we pick two times two card of one suit and 3 card of other 3 suits:
        \[ C_4^2 \cdot (C_{13}^2)^2 \cdot (C_{13}^2)^2 \]

        \[  C_4^1 \cdot C_{13}^3 \cdot (C_{13}^1)^3 +  C_4^2 \cdot (C_{13}^2)^2 \cdot (C_{13}^2)^2 \]

    \item Set of such combinations:
        \[ D = \set{ a b c d e f :
            a, b, c, d, e \text{ -- cards of one suit} } \]

        The number of ways to pick \( a, b, c, d, e \) of one specific suit is \( C_{13}^5 \).
        Taking in consideration number of suits to pick:
        \[ 4 \cdot C_{13}^5 \]

        And last, last 6-th card should be of another 3 suits. Number of card in these suits --
        \( 13 \cdot 3 = 39 \). So number of variants to pick one card of 39 is \( C_{39}^1 = 39 \).

        Answer:
        \[ 4 \cdot C_{13}^5 \cdot C_{39}^1 \]

\end{enumerate}

%%%%%%%%%%%%%%%%%%%%%%%%%%%%%%%%%%%%%%%%%%%%%%%%%%%%%%%%%%%%%%%%%%%%%%%%%%%%%%%%%%%%%%%%%%%%%%%%
\exercise*[1.20]

\begin{enumerate}[label=\alph*]

    \item Use common approach to such problem. Instead assignment of letters to mailboxes
        lets place mail in one line and demarcate them by sticks. We use 3 sticks in order to
        produce 4 mailboxes. Is such a way every box should have at least 1 letter. The number of
        pickings a place for stick is (we got 9 places to place a stick) -- number of assignments:
        \[ C_9^3 \]

    \item C_9^2

\end{enumerate}

%%%%%%%%%%%%%%%%%%%%%%%%%%%%%%%%%%%%%%%%%%%%%%%%%%%%%%%%%%%%%%%%%%%%%%%%%%%%%%%%%%%%%%%%%%%%%%%%
\exercise*[1.21]

\begin{enumerate}[label=\alph*]

    \item Set of vouchers:
        \[ V = \set{ 1, 2, \dots, 10 } \]
        
        Create set of permutations (just mapping a voucher to some student or vice versa):

        \[  P = \set{
            \begin{pmatrix}
                1   & 2   & \dots & 10 \\
                v_1 & v_2 & \dots & v_{10}
            \end{pmatrix} :
            v_i \in V
        }\]

        The number of such permutations is:
        \[ \abs P = 10! \]

    \item So some of the vouchers are the same. Define a set of such assignments:
        \[  A = \set{
            \begin{pmatrix}
                1   & 2   & \dots & 10 \\
                v_1 & v_2 & \dots & v_{10}
            \end{pmatrix} \; | \;
            v_i \text{ -- 1-st type } \, i=\overline{1, 4}, \; \\
            v_j \text{ -- 2-nd type } \, i=\overline{5, 10}
        }\]

        Here we firstly assign 4 vouchers of 1-st type to some of \( 10 \) students:
        \[ C_{10}^{4} \]
        And then assign last 6 vouchers of 2-nd type to \( 10 - 4 \) left students:
        \[ C_6^6 \]

        The number of such assignments is:
        \[ C_{10}^{4} \cdot C_6^6 = \frac{10!}{4! 6!} \]
        

\end{enumerate}


%%%%%%%%%%%%%%%%%%%%%%%%%%%%%%%%%%%%%%%%%%%%%%%%%%%%%%%%%%%%%%%%%%%%%%%%%%%%%%%%%%%%%%%%%%%%%%%%
\exercise*[1.22]

In order to get from \( (0, 0, \dots, 0) \) to \( (n_1, n_2, \dots, n_r) \) \( n_i \) steps along
each axis \( i \) is required. So the only thing that changes along all the configurations is the order of
such steps. But all the steps along some axis are the same, the steps of axis \( i \) and axis \( j \)
differ because they are steps along different axes.

So we need to find a cardinality of the set:
\[ S = \set{ \set{ s_1, s_2, \dots, s_{\,\sum n_i} } : s_i \in \set{0, 1, \dots, r} \text{ -- is a step along some
axis represented by number} } \]
Also there must be exactly \( n_i \) steps along axis \( i \).

The number of such configurations is:
\[ \left( \sum_{i=1}^{r} n_i \right)! \]

But take in consideration that steps along some axis \( i \) do not differ. The number of
permutations of these steps along axis \( i \) is \( n_i! \) so devide number of all
configurations by these number of permutations in order to not count these configurations.

\[ \frac{ \sum_{i=1}^{r} n_i }{ \prod_{i=1}^{r} n_i! } \]

%%%%%%%%%%%%%%%%%%%%%%%%%%%%%%%%%%%%%%%%%%%%%%%%%%%%%%%%%%%%%%%%%%%%%%%%%%%%%%%%%%%%%%%%%%%%%%%%
\exercise*[1.23]

Define such sets:
\[ A = \set{ a : a \text{ is a student} } \]
\[ E = \set{ e : e \text{ is a student and knows english} } \]
\[ G = \set{ g : g \text{ is a student and knows german} } \]
\[ F = \set{ f : f \text{ is a student and knows french} } \]

Cardinality of the sets and their intersections is given:
\[ \abs A = 100, \;\; \abs E = 28, \;\; \abs G = 30; \;\; \abs F = 42 \]
\[ \abs{E \cap G} = 8; \;\; \abs{E \cap F} = 10; \;\; \abs{G \cap F} = 5; \;\; \abs{E \cap G \cap F}
= 3\]

In order to use \textbf{Inclusion–exclusion principle} set of students that do not know neither
english nor german nor french.
\[ N = \set{ n : n \text{ is a student and knows neither english nor german nor french} } \]

Use \textbf{Inclusion–exclusion principle} taking in consideration that set \( N \)
does not intersects with other language-learners sets:
\[ \abs A = \abs E + \abs G + \abs F + \abs N - \abs{E \cap G} - \abs{E \cap F} - \abs{G \cap F} +
\abs{E \cap G \cap F} \]
\[ \abs N = \abs A - \abs E - \abs G - \abs F + \abs{E \cap G} + \abs{E \cap F} + \abs{G \cap F} -
\set{E \cap G \cap F} \]
\[ \abs N = 100 - 28 - 30 - 42 + 8 + 10 + 5 - 3 \]
\[ \abs N = 20 \]


\end{document}

