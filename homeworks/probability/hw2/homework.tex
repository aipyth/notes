\documentclass{homework}
\usepackage{enumitem}

\title{Probability Theory Homework №2}
\author{Ivan Zhytkevych}

\begin{document}

\maketitle

\exercise*[3.15]
\textit{Prove that for arbitrary events A and B:}
\[ P(AB) = 1 - P(\overline A) - P(\overline B) + P(\overline A \,  \overline B) \]
\textbf{Solution:}
\[ P(AB) = P(A) + P(B) - P(A \cup B) = \]
\[ = 1 - P(\overline A)) + (1 - P(\overline B)) - (1 - P(\overline{A \cup B})) = \]
\[ = 1 - P(\overline A) - P(\overline B) + P(\overline A \, \overline B)) \]

%%%%%%%%%%%%%%%%%%%%%%%%%%%%%%%%%%%%%%%%%%%%%%%%%%%%%%%%%%%%%%%%%%%%%%%%%%%%%%
\exercise*[3.16]
\textit{Four dice are tossed. Find the probability that they will lose the same number of points.}

\textbf{Solution:}

Set of all dice configurations:
\[ \Omega = \set{(a_1, a_2, a_3, a_4) : a_i = \overline{1, 6}} \]
\[ \abs \Omega = 6^4 \]

The configurations we're interested in:
\[ S = \set{(a_1, a_2, a_3, a_4) : a_1 = a_2 = a_3 = a_4, \; a_i = \overline{1,6}} \]
\[ \abs S = 6 \]

All configurations are equiprobable:
\[ P(S) = \frac{\abs S}{\abs \Omega} = \frac{6}{6^4} = \frac{1}{6^3} \]


%%%%%%%%%%%%%%%%%%%%%%%%%%%%%%%%%%%%%%%%%%%%%%%%%%%%%%%%%%%%%%%%%%%%%%%%%%%%%%
\exercise*[3.17]
\textit{Class consists of r students. Find the probability that at least two students were born in
the same month (assume all months are equiprobable for birth).}

\textbf{Solution:}

Make a set of all birth configurations:
\[ \Omega = \set{ (a_1, a_2, \dots, a_{12}) : a_i \text{ -- students born in i-th month}, \;
\sum a_i = r } \]

To find probability that at least two students were born in the same month let's find
probability of opposite event: number of configurations where none of the students were born in the
same month.
If \( r \) gets bigger than the number of month our probability inevitably turns into 0 because
there's no such configurations where each month has less than 2 birthday boy or girl. So we found
the probability we need for one case -- \( r > 12 \):
\[ r>12: P = 1 \]

To find the probability for \( r < 12 \) we follow this path of thinking:

The probability that at least two students have the same birthday month is opposite to probability
that none of them has same birthday month. This probability is easier to find:
\[ A = \set{ (a_1, a_2, \dots, a_{12}) : a_i < 2, \; \sum a_i = r} \]
\[ \abs A = A_{12}^{r} \]
\[ \overline P = \frac{ \abs A }{12^r} = \frac{A_{12}^{r}}{12^r} \]

\[ P = 1 - \frac{A_{12}^{r}}{12^r} \]

%%%%%%%%%%%%%%%%%%%%%%%%%%%%%%%%%%%%%%%%%%%%%%%%%%%%%%%%%%%%%%%%%%%%%%%%%%%%%%
\exercise*[3.18]
\textit{How many people should be in the room for the probability that at least two of those present
were born on the same day was more than half?}

\textbf{Solution:}

\( n \) -- number of people in the room.

Using the same approach as in the previous problem: n should be less than number of days in the
year in order to calculate the probability we want. Let the number of days in the year be 365, leap year is not taken into consideration.
\[ \Omega = \set{ (a_1, a_2, \dots, a_{365}) : a_i = \overline{1, 365}, \; \sum a_i = n } \]

The probability is calculated the same way as in the previous task:
\[ P = 1 - \frac{A_{365}^{n}}{ 365^n } = 1 - \frac{ 365! }{ (365 - n)! \cdot 364^n } \]

Found number we need using binary search for a range from 2 to 50:

\[ n = 23 : P = 1 - \frac{365!}{(365 - 23)! \cdot 365^{23}} \approx 0.507 \]



%%%%%%%%%%%%%%%%%%%%%%%%%%%%%%%%%%%%%%%%%%%%%%%%%%%%%%%%%%%%%%%%%%%%%%%%%%%%%%
\exercise*[3.19]
\textit{Numbers 1, 2, \dots, n are in random order. Find probability that numbers: a) 1 and 2; b) 1,
2 and 3 are placed next to in the predefined order.}

\textbf{Solution:}

\[ N = \set{1, 2, \dots, n} \]
Set of all such number configurations:
\[ A = \set{ (a_1, a_2, \dots, a_n) : a_i \in N \backslash \set{a_{i-1}, \dots, a_1} } \]
\[ \abs A = n! \]

\begin{enumerate}
    \item \textbf{1 and 2.}
        \textit{D} is our event that number 2 is next to 1. The cardinality of this set can be
        calculated in the following way: number of positions to pick for 1 is \(n-1\) since number 2
        is placed next to. Time to arrange other \(n-2\) number in \( (n-2)! \) ways.
        \[ D = \set{ t = (a_1, a_2, \dots, a_n) : t \in A, \; a_i = 1, a_{i+1} = 2 } \]
        \[ \abs D = (n-1) \cdot (n-2)! \]

        The probability is:
        \[ P(D) = \frac{ (n-1)(n-2)! }{ n! } = \frac{ (n-1)! }{ n! } \]

    \item \textbf{1, 2 and 3.}
        Use the same approach. \textit T is our event that \textit{1, 2 and 3} are placed next to
        each other. The number of such positions is \( n - 2 \).
        \[ T = \set{ t = (a_1, a_2, \dots, a_n) : t \in A, \; a_i = 1, \, a_{i+1} = 2,\, a_{i+2} = 3
        }\]
        \[ \abs T = (n-2) \cdot (n-3)! \]
        
        The probability is:
        \[ P(T) = \frac{ (n-2)! }{ n! } \]

\end{enumerate}


%%%%%%%%%%%%%%%%%%%%%%%%%%%%%%%%%%%%%%%%%%%%%%%%%%%%%%%%%%%%%%%%%%%%%%%%%%%%%%
\exercise*[3.20]
\textit{Exam consists of 10 questions. Each question should be answeres "yes" or "no". Find a
probability that student will give correct answers to at least 70\% of all questions picking answer
by chance. Solve this problem for a test of 30 and 50 questions.}

\textbf{Solution:}

\[ \Omega = \set{ (\omega_1, \dots, \omega_{10}) : \omega_i \in \set{0, 1} } \]
\[ \abs \Omega = 2^{10} \]

Our event that student answered at least 70\% of questions:
\[ T = \set{ \omega \in \Omega : \sum \omega_i \geq 7 } \]

There are up to 3 of these questions can remain answered falsly. Pick answers that will be true.
\[ \abs T = \sum_{i=7}^{10} C_{10}^{i} \]

The probability:
\[ P(T) = \frac{ \sum_{i=7}^{12} C_{10}^{i} }{2^{10}} \]

Generalizing for 30 questions and 50:

Number of true answers for 30 and 50 respectively is \( > 21 \) and \( > 35 \).

\[ P(T_{30}) = \frac{ \sum_{i=21}^{30} C_{30}^{i} }{ 2^{30} } \]
\[ P(T_{50}) = \frac{ \sum_{i=35}^{50} C_{50}^{i} }{ 2^{50} } \]


%%%%%%%%%%%%%%%%%%%%%%%%%%%%%%%%%%%%%%%%%%%%%%%%%%%%%%%%%%%%%%%%%%%%%%%%%%%%%%
\exercise*[3.21]
\textit{A list of 4N tournament participants is divided into four parts. Find a probability that
four strongest participants will be in distinct groups.}

\textbf{Solution:}
To divide participants into 4 groups firstly pick \textit N participants for first group, then pick
\textit N for the second out of \textit{3N} participants left and so on.
\[ \abs \Omega = C_{4N}^{N} \cdot C_{3N}^{N} \cdot C_{2N}^{N} \]
Find all groups where first strongest participants are in distinct group: 

4 possible variants to assign group for first participant

3 possible variants to assign group for second participant and so on...

After assigning strongest participants to their groups other \textit{4N - 4} ones are left:

\[ \abs A = 4! ( C_{4N-4}^{N-1} C_{3N_3}^{N-1} C_{2N-2}^{N-1} ) \]

A is a set of all such possible groups. At the end the probability is:
\[ P(A) = \frac{ 4! ( C_{4N-4}^{N-1} C_{3N_3}^{N-1} C_{2N-2}^{N-1} ) }{ C_{4N}^{N} C_{3N}^{N}
C_{2N}^{N} } \]


%%%%%%%%%%%%%%%%%%%%%%%%%%%%%%%%%%%%%%%%%%%%%%%%%%%%%%%%%%%%%%%%%%%%%%%%%%%%%%
\exercise*[3.22]
\textit{Playing poker, a player has five cards from a deck of 52 cards. Find the probability of the
following combinations:}
\begin{enumerate}[label=\alph*]
    \item
        \textbf{royal flush} ten, jack, queen, king and ace of the same suit
    \item 
        \textbf{straight flush} five cards in a row of the same suit but not royal flush
    \item
        \textbf{four of a kind} four cards of the same order
    \item
        \textbf{full house} two and three cards of the same order respectively
    \item
        \textbf{flush} five cards of the same suit but neither royal flush nor straight flush
    \item
        \textbf{straight} five card in a row, not all are of the same suit
\end{enumerate}

\textbf{Solution:}

\[ \Omega = \set{ \set{a_1, a_2, \dots, a_5} : a_i \text{ -- card from the deck} } \]
\[ \abs \Omega = C_{52}^{5} \]

\begin{enumerate}[label=\alph*]
    \item \textbf{royal flush}
        \[ A = \set{ f =  \set{a, b, c, d, e} : \\
                \text{f is a royal flush combination} \\ 
        } \]
        \[ \abs A = 4 \]
        \[ P(A) = \frac{4}{C_{52}^{5}} \]


    \item \textbf{straight flush}
        Straight flush configurations are all configurations of 13 cards of the same suit of
        sequential rank except \textbf{royal flush}.
        \[ B \text{ -- straight flush configurations}  \]
        \[ \abs B = 4(13 - 1) \]

        \[ P(B) = \frac{ 4(13 - 1) }{ C_{52}^{5} } = \frac{ 32 }{ C_{52}^{5} } \]

    \item \textbf{four of a kind}
        Four of a kind does not intersects with royal flush or straight flush.
        13 variants to pick four cards of one rank out of 4 suits plus one card of some other rank.
        \[ \abs{C} = 13 \cdot C_{48}^{1} \]

        \[ P(C) = \frac{13 \cdot 48}{C_{52}^{5}} \]

    \item \textbf{full house}
        Firstly pick a rank then a suit for 3 and 2 cards respectively.
        \[ 2: 13 \cdot C_{4}^{2}  \]
        \[ 3: 12 \cdot C_{4}^{3} \]
        Apply product rule:
        \[ \abs{D} = 13 \cdot C_{4}^{2} \cdot 12 \cdot C_{4}^{3} = 156 \cdot 24 = 3744 \]
        \[ P(D) = \frac{3744}{C_{52}^{5}} \]


    \item \textbf{flush}
        \[ \abs E = 4(C_{13}^{5} - 8 - 1) = 4(117 - 9) = 4 \cdot 108 = 432 \]
        \[ P(E) = \frac{432}{C_{52}^{5}} \]

    \item \textbf{straight}
        Fistly pick a sequential rank configuration. 13 possible variants to make such
        configuration. Then pick suits, but there are 5 possible ways to pick them:
        \[ 1, 1, 2, 3, 4 \; : \;
            4 \text{ combinations} \]
        \[ 1, 1, 2, 2, 3 \; : \;
            \frac{4*3}{2} \cdot 2 = 12 \text{ combinations}\]
        \[ 1, 1, 1, 2, 2 \; : \;
            4 * 3 = 12 \text{ combinations}\]
        \[ 1, 1, 1, 2, 3 \; : \;
            4 * \frac{3*2}{2} = 12 \text{ combinations}\]
        \[ 1, 1, 1, 1, 2 \; : \;
            4 * 3 = 12 \text{ combinations}\]
        All in all:
        \[ \abs{F} = 13 ( 4 + 4 * 12 ) = 676 \]
        \[ P(F) = \frac{676}{C_{52}^{5}} \]
        

\end{enumerate}


%%%%%%%%%%%%%%%%%%%%%%%%%%%%%%%%%%%%%%%%%%%%%%%%%%%%%%%%%%%%%%%%%%%%%%%%%%%%%%
\exercise*[3.23]
\textit{Computer center has 3 processors that received n tasks. For each task the processor is
randomly picked. Find a probability that exactly one processor will be idle.}

\textbf{Solution:}

\[ \Omega = \set{ ( a_1, a_2, \dots, a_n ) : a_i = \overline{1, 3} } \]
\[ \abs \Omega = 3^n \]

The cases we're interested in:
\[ A_i = \set{ (a_1, a_2, \dots, a_n ) : a_i \in \set{1, 2, 3} \backslash \set{i} } \]
\[ \abs{A_i} = 2^n \]
\[ A = A_1 + A_2 + A_3 \]
\[ P(A) = \frac{ 3 * 2^n }{ 3^n } = \frac{2^n}{3^{n-1}} \]


\end{document}

