% arara: pdflatex
\documentclass[12pt,letterpaper]{article}
\usepackage{fullpage}
\usepackage[top=2cm, bottom=4.5cm, left=2.5cm, right=2.5cm]{geometry}
\usepackage{amsmath,amsthm,amsfonts,amssymb,amscd}
\usepackage{hyperref}
% \usepackage{xcolor}
\usepackage[dvipsnames]{xcolor}
\usepackage{fancyhdr}
\usepackage{mathrsfs}
\usepackage{amsmath}

\usepackage{mathtools}
\DeclarePairedDelimiter{\ceil}{\lceil}{\rceil}
\DeclarePairedDelimiter{\set}{\left\{}{\right\}}

\usepackage{fontspec}

\setromanfont{PTSerif}[
    Path=./fonts/,
    Extension = .ttf,
    UprightFont = *-Regular,
    BoldFont = *-Bold,
    ItalicFont = *-Italic,
    BoldItalicFont = *-BoldItalic,
]

\hypersetup{%
  colorlinks=true,
  linkcolor=blue,
  linkbordercolor={0 0 1}
}

%%%%%%%%%%%%%%%%%%%%%%%%%%%%%%%%%%%%%%%%%%%%%%%%%%%%%%%%%%%%%%%%%%%%%%%%%%%%%%
% Define Colors
%%%%%%%%%%%%%%%%%%%%%%%%%%%%%%%%%%%%%%%%%%%%%%%%%%%%%%%%%%%%%%%%%%%%%%%%%%%%%%
\definecolor{light-gray}{gray}{0.85}

%%%%%%%%%%%%%%%%%%%%%%%%%%%%%%%%%%%%%%%%%%%%%%%%%%%%%%%%%%%%%%%%%%%%%%%%%%%%%%
\newcommand\hwnumber{8}
\newcommand\student{Ivan Zhytkevych}

%%%%%%%%%%%%%%%%%%%%%%%%%%%%%%%%%%%%%%%%%%%%%%%%%%%%%%%%%%%%%%%%%%%%%%%%%%%%%%
\pagestyle{fancyplain}
\headheight 35pt
\lhead{\textit{\student}}
\chead{\textbf{\Large PT Homework \hwnumber}}
\rhead{\textit{\today}}
\lfoot{}
\cfoot{}
\rfoot{\small\thepage}
\headsep 1.5em

%%%%%%%%%%%%%%%%%%%%%%%%%%%%%%%%%%%%%%%%%%%%%%%%%%%%%%%%%%%%%%%%%%%%%%%%%%%%%%
\begin{document}
% \tableofcontents
% \newpage
%%%%%%%%%%%%%%%%%%%%%%%%%%%%%%%%%%%%%%%%%%%%%%%%%%%%%%%%%%%%%%%%%%%%%%%%%%%%%%%
\section*{Problem 7.13}

\noindent\fcolorbox{black}{light-gray}{
    \parbox{\textwidth}{
        Опишіть $\sigma$-алгебру підмножин відрізка $[0, 1]$, породжену
        множинами:
        \begin{enumerate}
            \item $ \left[ \frac{1}{3}, \frac{1}{2} \right] $
            \item множиною раціональних точок відрізка $[ 0, 1 ]$
            \item $\{0\}$ та $\{1\}$
        \end{enumerate}
    }
}
%%%%%%%%%%%%%%%%%%%%%%%

\begin{enumerate}
    \item $ [ \frac{1}{3}, \frac{1}{2} ] $
        \[ H_1 = \left[ 0, \frac{1}{3} \right) \]
        \[ H_2 = \left[ \frac{1}{3}, \frac{1}{2} \right] \]
        \[ H_3 = \left( \frac{1}{2}, 1 \right] \]

        \begin{gather*}
            \mathcal{F}_{[1/3, 1/2]} = \sigma\left(\left[ \frac{1}{3},
            \frac{1}{2} \right]\right) = \{
                \emptyset, \Omega,
                \left[ 0, \frac{1}{3} \right),
                \left[ \frac{1}{3}, \frac{1}{2} \right],
                \left( \frac{1}{2}, 1 \right], \\
                \left[ 0, \frac{1}{3} \right) \cup \left[ \frac{1}{3}, \frac{1}{2} \right],
                \left[ 0, \frac{1}{3} \right) \cup \left( \frac{1}{2}, 1 \right],
                \left[ \frac{1}{3}, \frac{1}{2} \right] \cup \left( \frac{1}{2}, 1 \right]
            \}
        \end{gather*}

    \item множина раціональних точок відрізка $[0, 1]$

        \[ A = \set{ \frac{q_1}{q_2} : q_i \in \mathbb{N}, q_1 \leq q_2 } \]

        \[ H_1 = A \]
        Введемо $H_2$, що буде містити всі інтервали, котрі не містять
        раціонального числа, тобто це множина всіх ірраціональних чисел.
        \[ H_2 = [0, 1] \backslash A \]

        \[ \mathcal{F}_{A} = \set{ \emptyset, \Omega, A, H_2 } \]

    \item $\{0\}$ та $\{1\}$

        \[ H_1 = (0, 1) \]
        \[ H_2 = \set{0} \]
        \[ H_3 = \set{1} \]
    \[ \mathcal{F}_{\{0\}, \{1\}} = \set{ \emptyset, \Omega, H_1, H_2, H_3,
        H_1 \cup H_2, H_1 \cup H_3, H_2 \cup H_3 } \]
    
\end{enumerate}

%%%%%%%%%%%%%%%%%%%%%%%%%%%%%%%%%%%%%%%%%%%%%%%%%%%%%%%%%%%%%%%%%%%%%%%%%%%%%%%
\section*{Problem 7.14}

\noindent\fcolorbox{black}{light-gray}{
    \parbox{\textwidth}{
        Нехай $\Omega = \mathbb{R}$, $\mathcal{B}$ - борелева $\sigma$-алгебра
        на $\mathbb{R}$. Доведіть, що
        $\mathbb{R} \backslash \mathbb{Q} \in \mathcal{B}$.
    }
}
%%%%%%%%%%%%%%%%%%%%%%%

\[ \mathcal{B}(\mathbb{R}) \]
\[ \mathbb{R} \backslash \mathbb{Q} = \overline{\mathbb{Q}} \;\;
    (\mathbb{Q} \subset \mathbb{R})\]

Візьмемо такий інтервал, що, очевидно, буде належати нашій алгебрі:
\[ [q, q] = { q \in \mathbb{Q} } \]
Об'єднання таких інтервалів буде належати алгебрі.
\[ \bigcup\limits_{q \in \mathbb{Q}} [q,q] \in \mathcal{B}(\mathbb{R}) \Rightarrow \]
\[ \bigcup\limits_{q \in \mathbb{Q}} [q,q] = \mathbb{Q} \in \mathcal{B}(\mathbb{Q}) \]
А отже і за означенням алгебри маємо:
\[ \overline{\mathbb{Q}} = \mathbb{R} \backslash \mathbb{Q} \in
    \mathcal{B}(\mathbb{R})\]

%%%%%%%%%%%%%%%%%%%%%%%%%%%%%%%%%%%%%%%%%%%%%%%%%%%%%%%%%%%%%%%%%%%%%%%%%%%%%%%
\section*{Problem 7.15}

\noindent\fcolorbox{black}{light-gray}{
    \parbox{\textwidth}{
        Нехай $\Omega = \mathbb{R}^2$, $\mathcal{B}$ - борелева $\sigma$-алгебра
        в $\mathbb{R}^2$. Доведіть, що
        \begin{enumerate}
            \item $ (\mathbb{R}\backslash\mathbb{Q}) \times \mathbb{Q}
                \in \mathcal{B} $
            \item $ \set{ (x_1,x_2) \in R^2 \mid 
                    \max ( \sin (x_1 x_2), \arctan (x_2 - x_1)) > 0.1
                } \in \mathcal{B} $
        \end{enumerate}
    }
}
%%%%%%%%%%%%%%%%%%%%%%%

\begin{enumerate}
    \item З минулої задачі довели, що множина раціональних чисел належить
        борелевій алгебрі над раціональними числами, а отже і
        $\mathbb{R}\backslash\mathbb{Q} \in \mathcal{B}$.
        А декартовий добуток борелевих множин є також борелевою множиною.
        Тому і $ (\mathbb{R}\backslash\mathbb{Q}) \times \mathbb{Q}
                \in \mathcal{B} $.

            \item sin та arctan є неперевними функціями, а отже і max(sin, arctan)
                є неперервною. Оскільки $\max(...) > 0.1$ є не що інше, як
                $\max(...) \in (0.1, \infty)$, то його образ є відкритою множиною.
                Праобраз відкритої множини --- множина відкрита. Отже дана множина
                є відкритою та належить борелевій алгебрі.
\end{enumerate}

%%%%%%%%%%%%%%%%%%%%%%%%%%%%%%%%%%%%%%%%%%%%%%%%%%%%%%%%%%%%%%%%%%%%%%%%%%%%%%%
\section*{Problem 7.17}

\noindent\fcolorbox{black}{light-gray}{
    \parbox{\textwidth}{
        Опишіть $\sigma$-алгебру, породжену:
        \begin{enumerate}
            \item подіями нульової ймовірності
            \item подіями ймовірності одиниця.
        \end{enumerate}
    }
}
%%%%%%%%%%%%%%%%%%%%%%%

\[ \mathbb{A} = \set{ A_i : P(A_i) = 0 }; \;\; |\mathbb{A}| = n \]

\begin{enumerate}
    \item $ \sigma (A_i) $
        \[ P(\overline{A_i}) = 1 \]
        \[ P\left(\bigcup_i  A_i \right) = \sum_i P(A_i) = 0 \]
        \[ P\left(\bigcup_i \overline A_i \right) = \sum_i P(\overline A_i) 
        - \sum_{i\neq j} P(\overline A_i \overline A_j) + \dots
        = 1 \]
        \[ \bigcup_i A_i \cup \bigcup_i \overline A_i = \Omega \]
        \[ P(\Omega) = 1 \]
    \item $ \sigma (\overline{A}_i) $ аналогічно $\sigma(A_i)$:
        \[ \bigcap_i \overline{A}_i = \Omega \backslash \bigcup_i A_i \]
\end{enumerate}

%%%%%%%%%%%%%%%%%%%%%%%%%%%%%%%%%%%%%%%%%%%%%%%%%%%%%%%%%%%%%%%%%%%%%%%%%%%%%%%
\section*{Problem 7.18}

\noindent\fcolorbox{black}{light-gray}{
    \parbox{\textwidth}{
        Нехай $\{ A_n \}_{n \geq 1}$ - деяка послідовність подій. Доведіть,
        що подія $\underline{\lim}A_n$ належить $\sigma$-алгебрі, породженій
        цією послідовністю.
    }
}
%%%%%%%%%%%%%%%%%%%%%%%

\[ \underline{\lim} A_i = \bigcup_{i=1}^{\infty} \bigcap_{j=i}^{\infty} A_i \]
\[ \mathbb{A} = \sigma (A_n) \Rightarrow A_n \in \mathbb{A}; \;\;\;
    \bigcup_j A_j \in \mathbb{A}, \; j \geq 1; \;\;\;
    \bigcap_j A_j \in \mathbb{A}, \; j \geq 1 \]
\[ \Rightarrow \underline{\lim} A_n \in \mathbb{A} \]

%%%%%%%%%%%%%%%%%%%%%%%%%%%%%%%%%%%%%%%%%%%%%%%%%%%%%%%%%%%%%%%%%%%%%%%%%%%%%%%
\section*{Problem 7.19}

\noindent\fcolorbox{black}{light-gray}{
    \parbox{\textwidth}{
        Доведіть, що якщо $\mathcal{A}_1, \mathcal{A}_2, ...$ - неспадна
        послідовність $\sigma$-алгебр, то їхнє об'єднання $\mathcal{A} = 
        \bigcup\limits_{n=1}^{\infty} \mathcal{A}_n$ є алгеброю.
    }
}
%%%%%%%%%%%%%%%%%%%%%%%
\newline \\

Оскільки кожна алгебра містить як пусті множину, так і $\Omega$, то і об'єднання
цих алгебр буде містити $\emptyset$ та $\Omega$.

Оскільки послідовність є неспадною $\mathcal{A}_1 \subseteq \mathcal{A}_2
\subseteq... $, то отримана алгебра $ \mathcal{A} $ є найбільшою та містить
елементи всіх алгебр у послідовності.

Тому, якщо який елемент, або його заперечення, міститься у алгебрі $i$, то його заперечення, або ж він відповідно, міститься в цій алгебрі також, а отже і в
об'єднанні, тобто в алгебрі $\mathcal{A}$.

Якщо якесь об'єднання подій міститься у алгебрі $i$, при умові, що ці події
належать цій алгебрі, то і таке об'єднання буде належати і алгебрі $\mathcal{A}$.

%%%%%%%%%%%%%%%%%%%%%%%%%%%%%%%%%%%%%%%%%%%%%%%%%%%%%%%%%%%%%%%%%%%%%%%%%%%%%%%

\end{document}

