\documentclass{homework}
\usepackage{enumitem}
\usepackage{pgfplots}
\pgfplotsset{compat=1.15}
\usepackage{mathrsfs}
\usetikzlibrary{arrows}


\title{Probability Theory Homework №3}
\author{Ivan Zhytkevych}

\begin{document}

\maketitle

%%%%%%%%%%%%%%%%%%%%%%%%%%%%%%%%%%%%%%%%%%%%%%%%%%%%%%%%%%%%%%%%%%%%%%%%%%%%%%
\exercise*[4.20]
\textit{On a segment \(AB\) of length \textit l point O is chosen by chance. Find the probability that the
ratio \( \abs{AO}:\abs{AB} \) does not exceed 0.6 }

\textbf{Solution:}

\definecolor{xdxdff}{rgb}{0.49019607843137253,0.49019607843137253,1}
\definecolor{ududff}{rgb}{0.30196078431372547,0.30196078431372547,1}
\begin{tikzpicture}[line cap=round,line join=round,>=triangle 45,x=1cm,y=1cm]
    \clip(-10.5, -0.5) rectangle (1,1);
    \draw [line width=1pt] (-5,0)-- (0,0);
    \begin{scriptsize}
        \draw [fill=ududff] (-5,0) circle (2.5pt);
        \draw[color=ududff] (-5,0.3) node {$A$};
        \draw [fill=ududff] (0,0) circle (2.5pt);
        \draw[color=ududff] (0,0.3) node {$B$};
        \draw [fill=xdxdff] (-3,0) circle (2.5pt);
        \draw[color=xdxdff] (-3,0.3) node {$O$};
    \end{scriptsize}
\end{tikzpicture}

\[ \Omega = \set{ \frac{x}{y} : x = AO, y=AB } \]
\[ A = \set{ \frac{x}{y} \in \Omega : x = [0, 0.6] } \]



%%%%%%%%%%%%%%%%%%%%%%%%%%%%%%%%%%%%%%%%%%%%%%%%%%%%%%%%%%%%%%%%%%%%%%%%%%%%%%
\exercise*[4.21]
\textit{On a segment \([-1,2]\) two points A and B are chosen by chance. Fint the probability that
the distance between them is less then 2 and point A is to the left side of 1. }

\textbf{Solution:}


%%%%%%%%%%%%%%%%%%%%%%%%%%%%%%%%%%%%%%%%%%%%%%%%%%%%%%%%%%%%%%%%%%%%%%%%%%%%%%
\exercise*[4.22]
\textit{Point \( (\zeta, \mu) \) is randomly chosen in a square \( [0, 1]^2 \).
For a fixed \( z \in (0,1) \) find the probability: }

\begin{enumerate}
    \item \( P(\zeta + 2 \mu < z) \)
    \item \( P( max(\zeta, \mu) < z ) \)
    \item \( P( \zeta * \mu < z ) \)
\end{enumerate}

\textbf{Solution:}


%%%%%%%%%%%%%%%%%%%%%%%%%%%%%%%%%%%%%%%%%%%%%%%%%%%%%%%%%%%%%%%%%%%%%%%%%%%%%%
\exercise*[4.23]
\textit{}

\textbf{Solution:}



%%%%%%%%%%%%%%%%%%%%%%%%%%%%%%%%%%%%%%%%%%%%%%%%%%%%%%%%%%%%%%%%%%%%%%%%%%%%%%
\exercise*[4.24]
\textit{}

\textbf{Solution:}



%%%%%%%%%%%%%%%%%%%%%%%%%%%%%%%%%%%%%%%%%%%%%%%%%%%%%%%%%%%%%%%%%%%%%%%%%%%%%%
\exercise*[4.25]
\textit{}

\textbf{Solution:}



\end{document}

