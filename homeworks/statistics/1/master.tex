\documentclass[12pt,letterpaper]{article}
\usepackage{fullpage}
\usepackage[top=2cm, bottom=4.5cm, left=2.5cm, right=2.5cm]{geometry}
\usepackage{amsmath,amsthm,amsfonts,amssymb,amscd}
\usepackage{hyperref}
% \usepackage{xcolor}
\usepackage[dvipsnames]{xcolor}
\usepackage{fancyhdr}
\usepackage{mathrsfs}
\usepackage{amsmath}
\usepackage{dsfont}

\usepackage{mathtools}
\DeclarePairedDelimiter{\ceil}{\lceil}{\rceil}
\DeclarePairedDelimiter{\set}{\left\{}{\right\}}

\usepackage{fontspec}

\setromanfont{PTSerif}[
    Path=./fonts/,
    Extension = .ttf,
    UprightFont = *-Regular,
    BoldFont = *-Bold,
    ItalicFont = *-Italic,
    BoldItalicFont = *-BoldItalic,
]

\hypersetup{%
  colorlinks=true,
  linkcolor=blue,
  linkbordercolor={0 0 1}
}

%%%%%%%%%%%%%%%%%%%%%%%%%%%%%%%%%%%%%%%%%%%%%%%%%%%%%%%%%%%%%%%%%%%%%%%%%%%%%%
% Define Colors
%%%%%%%%%%%%%%%%%%%%%%%%%%%%%%%%%%%%%%%%%%%%%%%%%%%%%%%%%%%%%%%%%%%%%%%%%%%%%%
\definecolor{light-gray}{gray}{0.85}

%%%%%%%%%%%%%%%%%%%%%%%%%%%%%%%%%%%%%%%%%%%%%%%%%%%%%%%%%%%%%%%%%%%%%%%%%%%%%%
\newcommand\hwnumber{1}
\newcommand\student{Ivan Zhytkevych}

%%%%%%%%%%%%%%%%%%%%%%%%%%%%%%%%%%%%%%%%%%%%%%%%%%%%%%%%%%%%%%%%%%%%%%%%%%%%%%
\pagestyle{fancyplain}
\headheight 35pt
\lhead{\textit{\student}}
\chead{\textbf{\Large Statistics Homework \hwnumber}}
\rhead{\textit{\today}}
\lfoot{}
\cfoot{}
\rfoot{\small\thepage}
\headsep 1.5em

\usepackage{import}
\usepackage{xifthen}
\usepackage{pdfpages}
\usepackage{transparent}

\newcommand{\incfig}[1]{%
    \def\svgwidth{\columnwidth}
    \import{./figures/}{#1.pdf_tex}
}
\newcommand{\problemdef}[1]{%
    \noindent\fcolorbox{black}{light-gray}{
        \parbox{\textwidth}{
            #1
        }
    }
}


%%%%%%%%%%%%%%%%%%%%%%%%%%%%%%%%%%%%%%%%%%%%%%%%%%%%%%%%%%%%%%%%%%%%%%%%%%%%%%
\begin{document}
% \tableofcontents
% \newpage
%%%%%%%%%%%%%%%%%%%%%%%%%%%%%%%%%%%%%%%%%%%%%%%%%%%%%%%%%%%%%%%%%%%%%%%%%%%%%%%
\section*{Problem 17.12}

\problemdef{
    Нехай $\xi$ випадкова величина зі стандартним нормальним розподілом.
    Користуючись таблицею стандартного нормального розподілу знайдіть:
    \begin{enumerate}
        \item ймовірність $P(\xi \in [-3, 1])$
        \item таке $x$, що $P(|\xi| \leq x) = 0.9$.
    \end{enumerate}
}

\begin{enumerate}
    \item 
        \begin{gather*}
        P(\xi \in [-3, 1]) = P(\xi \leq 1) - P(\xi < -3) = \\
        = P(\xi \leq 1) - \left( 1 - P(\xi < 3)  \right) = \\
        = P(\xi \leq 1) + P(\xi < 3) - 1 = 0.841 + 0.999 - 1 = 0.84 
        \end{gather*}
    \item
        \begin{gather*}
            P(|\xi| \leq x) = P(\xi \leq x) - P(\xi \leq -x) = \\
            = P(\xi \leq x) - 1 + P(\xi \leq x) = \\
            = 2 P(\xi \leq x) - 1 = 0.9 \\
            P(\xi \leq x) = 0.95 \\
            x = 1.64
        \end{gather*}
\end{enumerate}

\[
\iff d = \frac{1009}{13}
.\] 

%%%%%%%%%%%%%%%%%%%%%%%%%%%%%%%%%%%%%%%%%%%%%%%%%%%%%%%%%%%%%%%%%%%%%%%%%%%%%%%
\section*{Problem 17.13}

\problemdef{
    Нехай $S$ - число гербів, що випали при 1 000 000 підкидань правильної монети.
    Оцініть ймовірність $(499 000 < S < 501 000)$:
    \begin{enumerate}
        \item за допомогою нерівності Чебишова
        \item за допомогою центральної граничної теореми
    \end{enumerate}
}

\begin{enumerate}
    \item 
        \begin{gather*}
            S = \sum_{i=1}^{10^6} \mathds{1}(\text{ випав герб }) \\
            P(S \geq \varepsilon) \leq \frac{M S}{\varepsilon} \\
            MS = 10^6 \cdot \frac{1}{2} = 5 \cdot 10^5 \\
            P(4.99 \cdot 10^5 < S < 5.01 \cdot 10^5) =
            P(S < 5.01 \cdot 10^5) - P(S < 4.99 \cdot 10^5) = \\
            = 1 - P(S \geq 5.01 \cdot 10^5) - 1 + P(S \geq 4.99 \cdot 10^5) = \\
            = P(S \geq 4.99 \cdot 10^5) - P(S \geq 5.01 \cdot 10^5) = \\
            \text{ навряд чи можна одразу перейти до нерівності } \\
            \text{ тому спробуємо інакше } \\
            P \left( 4.99 \cdot 10^5 < S < 5.01 \cdot 10^5 \right) = P(\left| S - 5 \cdot 10^5 \right| < 1000 ) = \\
            = 1 - P(\left| S - 5 \cdot 10^5 \right| \geq 1000 )\\
            \xi = \left| S - 5 \cdot 10^5 \right| = \left| S - M S \right|  \\
            \Rightarrow P(\left| S - M S \right| \geq 1000) \leq \frac{\mathcal{D}\xi}{\varepsilon^2} \\
            P(\left| S - M S \right| \geq 1000) \leq 10^6 \cdot \frac{1}{4} \cdot \frac{1}{10^6} \\
            P(\left| S - M S \right| \geq 1000) \leq \frac{1}{4}
            \Rightarrow P(\left| S - M S \right| < 1000 ) \leq \frac{3}{4}
        \end{gather*}
    \item 
        \begin{gather*}
            \frac{S - MS}{\sqrt{\mathcal{D}S}} = \frac{S - 5 \cdot 10^5}{500} \xrightarrow{d} N(0, 1) \\
            P(4.99 \cdot 10^5 < S < 5.01 \cdot 10^5) = \\
            = P(4.99 \cdot 10^5 - 5 \cdot 10^5 < S - 5 \cdot 10^5 < 5.01 \cdot 10^5 - 5 \cdot 10^5) = \\
            = P\left(\frac{4.99 \cdot 10^5 - 5 \cdot 10^5}{500} < \frac{S - 5 \cdot 10^5}{500} < \frac{5.01 \cdot 10^5 - 5 \cdot 10^5}{500}\right) = \\
            = P\left( -2 < \frac{S - 5 \cdot 10^5}{500} < 2 \right) = \\
            = P(\eta < 2) - P(\eta < - 2) = 2 P (\eta < 2) - 1 = 
            2 \cdot 0.977 - 1 = 0.954
        \end{gather*}
\end{enumerate}


%%%%%%%%%%%%%%%%%%%%%%%%%%%%%%%%%%%%%%%%%%%%%%%%%%%%%%%%%%%%%%%%%%%%%%%%%%%%%%%
\section*{Problem 17.15}
\problemdef{
    При випадковому блуканні на числовій осі частинка, стартуючи з точки 0, робить
    % крок вправо або вліво з ймовірністю \(\displaymath \frac{1}{2}$$. Оцініть
    ймовірність того, що після сотого кроку частинку віддалятиме від початкової
    точки не менше, ніж 10 кроків.


\begin{gather*}
    \xi = \begin{cases}
        1 & \frac{1}{2} \\
        -1 & \frac{1}{2}
    \end{cases} \\
    S_n = \sum_{i=1}^{n} \xi \\
    M S_n = 0 \\
    D S_n = 100 \\
    P(\left| S \right| < 10) = P(-10 < S < 10) = \\
    = P(\frac{-10}{\sqrt{100}} < \frac{S}{\sqrt{100} } < \frac{10}{\sqrt{100} }) = \\
    = P(-1 < \frac{S}{10} < 1) = P(\frac{S}{10} < 1) - P(\frac{S}{10} < -1) =
    P(\frac{S}{10} < 1) - 1 + P(\frac{S}{10} < 1) = \\
    = 2 P(\frac{S}{10} < 1) - 1 = 2 \cdot 0.841 - 1 = 0.682 \\
    \Rightarrow P(\left| S \right| > 10) = 0.318
\end{gather*}

%%%%%%%%%%%%%%%%%%%%%%%%%%%%%%%%%%%%%%%%%%%%%%%%%%%%%%%%%%%%%%%%%%%%%%%%%%%%%%%
\section*{Problem 17.16}
\problemdef{
    У компанії-авіаперевізнику оцінили, що в середньому 4\% квитків, проданих
    на авіарейс, залишаються невикористаними. Як наслідок, компанія обрала наступну стратегію: продавати 100 квитків на літак, що має 98 місць. Знайдіть ймовірність того, що кожному пасажиру, що опиниться на борту літака, знайдеться місце.
}

\begin{gather*}
    \xi_i = \mathds{1}(\text{ білет $i$ використаний })\\
    M \sum_{i=1}^{n} \xi_i = 0.04 \cdot n = \lambda \\ 
    \text{схоже на розподіл Пуассона, оскільки надали значення в середньому та не вказали кількість $n$} \\
    \mathcal{D} \sum_{i=1}^{n} \xi_i = \lambda \\
    S = \sum \xi_i \\
    \frac{S - M S}{\mathcal{D}S} = \frac{S - 0.04}{0.04} \to N(0, 1) \\
    \text{ нехай продано 100 білетів } \\
    P(S_{100} \geq 2) = P(S_{100} - 4 \geq 2 - 4) = P(\frac{S_{100} - 4}{\sqrt{4}} \geq \frac{2 - 4}{\sqrt{4}}) = P(\frac{S_{100} - 4}{2} \geq -1) =  \\
    = 1 - P(\frac{S_{100}-4}{2} < -1) = P(\frac{S_{100}-4}{2} < 1) = 0.841
\end{gather*}

%%%%%%%%%%%%%%%%%%%%%%%%%%%%%%%%%%%%%%%%%%%%%%%%%%%%%%%%%%%%%%%%%%%%%%%%%%%%%%%
\section*{Problem 17.17}
\problemdef{
    Кожен з членів жюрі, яке складається з непарної кількості осіб, незалежно
    від інших приймає справедливе рішення з ймовірністю $0.7$. Якою повинна бути
    чисельність жюрі для того, щоб рішення, ухвалене більшістю голосів, було
    справедливим з ймовірністю не меншою за $0.9$.
}

\begin{gather*}
    \eta = \mathds{1}( \text{ рішення справедливе }) \\
    S_n = \sum_{i=1}^{n} \eta_i \\
    M S_n = n \cdot 0.7 \\ 
    \mathcal{D}S_n = n \cdot 0.21 \\
    P(S_n > \frac{n}{2}) = P(\frac{S_n - n \cdot 0.7}{\sqrt{n \cdot 0.21}} > \frac{\frac{n}{2} - n \cdot 0.7}{\sqrt{ n \cdot 0.21}}) > 0.9 \\
    P(S_n > \frac{n}{2}) = 1 - P(S_n \leq \frac{n}{2}) \Rightarrow 
    1 - P(\frac{S_n - n \cdot 0.7}{\sqrt{n \cdot 0.21}} \leq \frac{\frac{n}{2} - n \cdot 0.7}{\sqrt{n \cdot 0.21}}) > 0.9 \\
    P(\frac{S_n - n \cdot 0.7}{\sqrt{ n \cdot 0.21}} \leq \frac{\frac{n}{2} - n \cdot 0.7}{\sqrt{ n \cdot 0.21}}) < 0.1 \\
    \frac{\sqrt{n} }{2 \cdot \sqrt{0.21}} - \frac{0.7 \cdot \sqrt{n} }{\sqrt{0.21}} < -1.282 \\
    \sqrt{n} \left( \frac{1}{2\cdot\sqrt{0.21}} - \frac{0.7}{\sqrt{0.21}}  \right) < -1.282 \\
    \sqrt{n} \left( 0.91 - 1.53 \right) < -1.282 \\
    \sqrt{n} > \frac{-1.282}{-0.62} \\
    \sqrt{n} > 2.07 \\
    n > 4.2
\end{gather*}

%%%%%%%%%%%%%%%%%%%%%%%%%%%%%%%%%%%%%%%%%%%%%%%%%%%%%%%%%%%%%%%%%%%%%%%%%%%%%%%
\section*{Problem 17.20}
\problemdef{
    Гральний кубик підкидається до тих пір, аж поки сума очок вперше перевищить $700$.
    Оцініть ймовірність того, що для цього знадобиться:
    \begin{enumerate}
        \item більше за 210 підкидань
        \item менше за 190 підкидань
        \item неменше за 180 і не більше за 210 підкидань.
    \end{enumerate}
}

\begin{gather*}
    \xi \text{ - кількість очок на кубику } \\
    M \xi = 3.5 \\
    \mathcal{D} \xi = \frac{1}{6}\left( 1 + 4 + 9 + 16 + 25 + 36 \right) - 3.5^2 = 2.91 \\
    S_n = \sum_{i=1}^{n} \xi_i \\ 
    M S_n = n \cdot 3.5 \\
    \mathcal{D}S_n = n \cdot 2.91 \\
\end{gather*}
\begin{enumerate}
    \item \begin{gather*}
    P(S_{210} < 700) = P(\frac{S_{210} - 210 \cdot 3.5}{\sqrt{210 \cdot 2.91}} < \frac{700 - 210 \cdot 3.5}{\sqrt{210 \cdot 2.91} }) = \\
    = P(\frac{S_{210} - 735}{24.72} < \frac{-35}{24.72}) = P(\eta < -1.42) = 0.078 \\
    \Rightarrow P(S_{>210} > 700) = 0.922
    \end{gather*}
    \item \begin{gather*}
        P(S_{190} > 700) = P(\frac{S_{190} - 190 \cdot 3.5}{\sqrt{190 \cdot 2.91}} > 1.49 ) = 1 - 0.932 = 0.068 \\
        \Rightarrow P(S_{<190} > 700) = 0.932
    \end{gather*}
\end{enumerate}

%%%%%%%%%%%%%%%%%%%%%%%%%%%%%%%%%%%%%%%%%%%%%%%%%%%%%%%%%%%%%%%%%%%%%%%%%%%%%%%%
%\section*{Problem 17.22}

%\problemdef{
%    Нехай $\{X_n\}_{n \geq 1}$ - послідовність незалежних однаково розподілених випадкових
%    величин з рівномірним розподілом на відрізку $[-3, 3]$ і нехай $T_n = \frac{1}{n} \sum_{i=1}^{n} \left| X_i \right|$. Знайдіть такі константи $a$ та $b^2$, що:
%    \[
%    \sqrt{n} \cdot (T_n - a) \xrightarrow{d} N(0, b^2)
%    .\] 
%}

%%%%%%%%%%%%%%%%%%%%%%%%%%%%%%%%%%%%%%%%%%%%%%%%%%%%%%%%%%%%%%%%%%%%%%%%%%%%%%%
\end{document}

